\appendix
\chapter{Supplementary Material}
\vspace{-\baselineskip}

\subsubsection*{Combinator Types (\textit{Teaspoon})}
\vspace{-0.5\baselineskip}

\begin{table}[H]
    \centering
    \begin{tabular}{lc||lll}
        \textbf{function} & \textbf{operator} & \textbf{LHS type} & \textbf{RHS type} & \textbf{combinator type} \\
        \hline
        \texttt{ap} & \texttt{\ap} & \texttt{P<(a: A) => B>} & \texttt{P<A>} & \texttt{P<B>} \\
        \texttt{pa} & \texttt{\pa} & \texttt{P<A>} & \texttt{P<(a: A) => B>} & \texttt{P<B>} \\
        \texttt{apL} & \texttt{\apl} & \texttt{P<A>} & \texttt{P<B>} & \texttt{P<A>} \\
        \texttt{apR} & \texttt{\apr} & \texttt{P<A>} & \texttt{P<B>} & \texttt{P<B>} \\
        \texttt{choice} & \texttt{\choice} & \texttt{P<A>} & \texttt{P<A>} & \texttt{P<A>} \\
        \texttt{mult} & \texttt{\mult} & \texttt{P<A>} & \texttt{P<B>} & \texttt{P<[A, B]>} \\
        \texttt{multL} & \texttt{\multl} & \texttt{P<A>} & \texttt{P<B>} & \texttt{P<A>} \\
        \texttt{multR} & \texttt{\multr} & \texttt{P<A>} & \texttt{P<B>} & \texttt{P<B>} \\
        \texttt{fmap} & \texttt{\fmap} & \texttt{(a: A) => B} & \texttt{P<A>} & \texttt{P<B>} \\
        \texttt{pamf} & \texttt{\pamf} & \texttt{P<A>} & \texttt{(a: A) => B} & \texttt{P<B>} \\
        \texttt{constFmapL} & \texttt{\constfmapl} & \texttt{A} & \texttt{P<B>} & \texttt{P<A>} \\
        \texttt{constFmapR} & \texttt{\constfmapr} & \texttt{P<A>} & \texttt{B} & \texttt{P<B>} \\
        \texttt{liftCons} & \texttt{<:>} & \texttt{P<A>} & \texttt{P<A[]>} & \texttt{P<A[]>} \\
        \texttt{label} & \texttt{<?>} & \texttt{P<A>} & \texttt{string} & \texttt{P<A>}
    \end{tabular}
    \caption{Implemented combinators and their respective types, \texttt{P} denotes \texttt{Parser} for brevity}
    \label{tab:combinators}
\end{table}

\subsubsection*{CLI Options}
\vspace{-0.5\baselineskip}

\begin{table}[H]
    \centering
    \begin{tabularx}{\linewidth}{ll||X}
        \textbf{abbreviation} & \textbf{type} & \textbf{description} \\
        \hline
        \texttt{-h} & $\times$ & Displays the help menu \\
        \texttt{-o} & file & (\textbf{required}) Path to target (new) file \\
        \texttt{-p} & toggle & Reformats the generated code (via the use of \textit{prettier}) \\
        \texttt{-a} & toggle & Enable all optimisations \\
        \texttt{-icc} & toggle & Implicitly convert chars / strings when used directly with a combinator \\
        \texttt{-cr} & toggle & Remove redundant alternatives \\
        \texttt{-lra} & toggle & Analyse and attempt remediation of left-recursive productions \\
        \texttt{-st} & toggle & Analyse string choice chains and optimise \\
        \texttt{-br} & toggle & Analyse attempt chains and optimise to reduce backtracking by factoring \\
        \texttt{-plo} & toggle & Basic optimisations using parser laws \\
        \texttt{-ngfs} & toggle & Restrict first set analysis to local declaration (global by default)
    \end{tabularx}
    \caption{Options and flags available for the CLI}
    \label{tab:cli_opt}
\end{table}

\subsubsection*{Example AST}
\begin{capminted}
    \begin{minted}{scala}
        BinaryOpExpr(
          BinaryOpExpr(
            BinaryOpExpr(
              Identifier("expr"),
              BinaryOpExpr(
                ArrowFunctionIdentifier(
                  Identifier("x"),
                  ArrowFunctionIdentifier(
                    Identifier("y"),
                    BinaryOpExpr(Identifier("x"), Identifier("y"), "+")
                  )
                ),
                MemberCall(
                  Identifier("chr"),
                  Arguments(None, List(Literal("'+'")))
                ),
                "<$"
              ),
              "<**>"
            ),
            Identifier("nat"),
            "<*>"
          ),
          Identifier("nat"),
          "<|>"
        )
    \end{minted}
    \vspace{-0.5\baselineskip}
    \caption{Parsed AST for running example of simple addition calculator}
    \label{lst:running_ast}
\end{capminted}