\section{Parsers}
\label{sec:parsers}

A parser forms the first stage of the compilation pipeline.
Intuitively, parsing (also referred to as syntactic analysis) is the process of obtaining meaning from an input, typically a string, and deriving meaning in some form of structure, based on some grammar.
For example, consider a simple expression: within these operators there are a series of precedence rules that should be adhered to, rules that are intuitively known.
However, this needs to be formally specified to a parser in order to generate the parse tree in \autoref{fig:tree}; the original input string has no real meaning without a structure.
\begin{figure}[H]
\centering
\begin{tikzpicture}
    \node[anchor=east] at (0, 0) {\texttt{-1+2*(3-4)}};
    \node at (1.4, 0) {$\leadsto$};
    \begin{scope}[shift={(4, 1.5)}]
        \node (o) at (0, 0) {\texttt{+}};
        \node (ol) at (-1, -1) {\texttt{-}};
        \node (or) at (1, -1) {\texttt{*}};
        \node (olc) at (-1, -2) {\texttt{1}};
        \node (orl) at (0.5, -2) {\texttt{2}};
        \node (orr) at (1.5, -2) {\texttt{-}};
        \node (orrl) at (1, -3) {\texttt{3}};
        \node (orrr) at (2, -3) {\texttt{4}};

        \draw
        (o) -- (ol)
        (o) -- (or)
        (ol) -- (olc)
        (or) -- (orl)
        (or) -- (orr)
        (orr) -- (orrl)
        (orr) -- (orrr);
    \end{scope}
\end{tikzpicture}
\caption{An example parse tree for a simple expression}
\label{fig:tree}
\end{figure}

It is important to first differentiate between the types of parsers.
In general, parsers can be split into two categories; top-down (including $\text{LL}(k)$ parsers, recursive descent) and bottom-up (including $\text{LR}(k)$ parsers, shift-reduce).
The former builds the AST from the root nodes to the leaf nodes, whereas the latter builds from the leaf nodes to the root nodes (hence bottom-up).

Top-down parsing begins with a non-terminal, and once a token is identified, it is used to `fill' in the tree, creating a root node and child nodes based on the expected structure of the root node from the first identified token.
At this point, any non-terminals that are in the tree are produced in the same way, building from the root node downwards.
Due to the recursive nature of this style of parsing, it is evident that recursively applying a rule can be problematic if care is not taken, as discussed in \autoref{ssec:lrec}.

On the other hand, bottom-up parsing tries to use all the rules (whereas top-down tries to match a non-terminal by trying each of the rules), and replaces it with a non-terminal, by using the production in reverse.
Bottom-up parsing succeeds when the whole input is replaced by the start symbol.

Both $\text{LL}(k)$ and $\text{LR}(k)$ parsers perform left-to-right scanning and have $k$ tokens of look-ahead \cite{dragon}, but differs in the former having left-most derivation whereas the latter has right-most derivation.
Note that $\text{LL}(n) \subseteq \text{LR}(n)$, hence $\text{LR}$ parsers are more powerful.
