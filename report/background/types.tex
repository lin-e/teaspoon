\section{Types}
\label{sec:type}

Throughout this section, the type \texttt{Parser} will be used.
In TypeScript, the host language for the project, this is defined as the following (where \texttt{Result} captures whether input is consumed, as well as whether the parser succeeded).

\begin{minted}{typescript}
type Parser<A> = (s: State) => Result<A>;
\end{minted}

However, for brevity and readability, Haskell's type syntax will be used instead.
Note that TypeScript does not have a specific type for characters, as such any use of \texttt{Char} in Haskell would be a use of \text{string} in TypeScript (however, a fixed length of 1 is assumed).
Note that additional work will also be required to concatenate a list of singleton strings into a single string, which will likely lead to the produced code being more verbose.

\subsection{Primitives}
\label{ssec:primitives}

As combinator parsing involves joining together basic primitives, it is natural to begin with the most basic parsers, as described by Hutton \& Meijer (1999) \cite{hutton99}.
The first primitive is that of a parser that will always succeed if it is able to consume input and fail otherwise.
This has the signature \texttt{item :: Parser Char}.

On the other hand, a parser that always fails, regardless of the input, is also required.
Note that this comes from the \texttt{Alternative} typeclass.
This parser can be referred to as \texttt{empty :: Parser a} - the type of which is irrelevant as no parse tree will be produced.

Finally, a primitive that lifts a pure value into the context of a parser is also required.
Note that this is in both the \texttt{Applicative} and \texttt{Monad} typeclasses, referred to as \texttt{pure} and \texttt{return} respectively.
As \texttt{Applicative} is a superclass of \texttt{Monad}, by default \texttt{return} refers to \texttt{pure}.
This has the following signature:

\begin{minted}{haskell}
return :: a -> Parser a
\end{minted}

Alongside \texttt{(>{}>=)} in \autoref{ssec:monad}, this can be used to form the parser \texttt{satisfy}, which parses a character based on a predicate.

While it is possible to define \texttt{satisfy} in terms of \texttt{item}, via the use of \texttt{bind}, it is also possible to go in the `opposite' direction by defining \texttt{item} as \texttt{satisfy (const true)}.
Using \texttt{satisfy} as the primitive is often preferable when compared to using \texttt{item} as \texttt{bind} significantly decreases the efficacy of inspection.
This primitive comes from a natural interpretation of reading a character and has the type:

\begin{minted}{haskell}
satisfy :: (Char -> Bool) -> Parser Char
\end{minted}

\subsection{Functor}
\label{ssec:functor}

A functor is a typeclass that allows for a function to be applied to the contained pure value, without altering the overall structure \cite{allen2016haskell}.
This is done with the following function:

\begin{minted}{haskell}
fmap :: (a -> b) -> Parser a -> Parser b
\end{minted}

It can also be written as \texttt{(<\$>)}, for infix notation.

By giving parsers this property, the user is now able to modify the resulting parse tree without changing any other properties of the parse result.
This includes the success of the result: if it was a failure, then nothing is done.
A concrete use of this is to create a parser for natural numbers: after consuming a sequence of characters (hence forming a string), it can be converted into an integer, giving a \texttt{Parser Int}.

Using the above, a derived combinator \texttt{(<\$)} arises.
The first argument is used in place of the resulting parse tree from the second argument (a \texttt{Parser}), if the parse was successful.
This has the signature:

\begin{minted}{haskell}
(<$) :: a -> Parser b -> Parser a
x <$ p = const x <$> p
\end{minted}

By treating \texttt{Parser}s as \texttt{Functor}s, it is possible to assign meaning to a result, by transforming it beyond that of a literal string.

\subsection{Applicative}
\label{ssec:applicative}

First note that an alternative typeclass to this is \texttt{Monoidal}, where both can capture the idea of sequencing parsers.
Earlier references, such as Wadler (1985) \cite{wadler85} and Fokker (1995) \cite{fokker95}, provide sequencing in the form of pairs:

\begin{minted}{haskell}
(<~>) :: Parser a -> Parser b -> Parser (a, b)
\end{minted}

While this may seem intuitive for simple combinations of parsers, it can become tedious and difficult to handle when more parsers are sequenced together.
For example, consider sequencing three parsers, leading to nested pairs:

\begin{minted}{haskell}
pa   :: Parser a
pb   :: Parser b
pc   :: Parser c
pabc :: Parser ((a, b), c)
  = pa <~> pb <~> pc
\end{minted}

In general use cases, the use of pairs to represent sequential operation can become too restrictive.
Naturally, this leads to a more general form.

Note that \texttt{Applicative}s allow for sequencing of functional computations but do not allow for the use of results from earlier computations (which are provided with \texttt{Monad} in \autoref{ssec:monad}).
They also do not generally enforce an execution order (however in the context of \texttt{Parser}s, this is enforced by \texttt{Monad}).
Recall the \texttt{return} `primitive' mentioned earlier in \autoref{ssec:primitives}, in the \texttt{Applicative} context, this is referred to as \texttt{pure}, however the semantics remain the same.
The method \texttt{(<*>)} can be intuitively thought of as a more general function application \cite{hutton16} in the context of \texttt{Parser}s:

\begin{minted}{haskell}
(<*>) :: Parser (a -> b) -> Parser a -> Parser b
\end{minted}

The aforementioned example of \texttt{(\mult)} can be written as follows (in terms of \texttt{<*>}) - note that the first parser is mapped into a parser of a function:

\begin{minted}{haskell}
px <~> py = (,) <$> px <*> py
\end{minted}

Important derived combinators that arise from this include the following \cite{yoda}:

\begin{minted}{haskell}
(<*)   :: Parser a -> Parser b -> Parser a
p <* q = const <$> p <*> q

(*>)   :: Parser a -> Parser b -> Parser b
p *> q = flip const <$> p <*> q

(<**>) :: Parser a -> Parser (a -> b) -> Parser b
p <**> q = (flip ($)) <$> p <*> q

(<:>)  :: Parser a -> Parser [a] -> Parser [a]
p <:> ps = (:) <$> p <*> ps
\end{minted}

The first two combinators execute both parsers, and only results in a successful parse when both succeed.
However, only one of the two parsed results are used, and the other is discarded.
The next combinator (\texttt{<**>}, or `reverse ap') is similar to the original \texttt{<*>}, but instead has the function as the second parser.
Finally, the \texttt{<:>} combinator, also referred to as lifted cons, allows joining an element to a list of elements in the \texttt{Parser} context.

In addition to these derived combinators, it is also possible to define `lift's, which combine $n$ parsers using a function.
Note that the derived combinators can also be defined in terms of lifts.
For example, \texttt{<:>} and \texttt{<**>} can be defined as the following (in TypeScript), where the first parameter is an uncurried function that contains the desired functionality:

\begin{minted}{typescript}
  let lift_cons = lift2((p: A, ps: A[]) => [p, ...ps], p, ps);
  let rev_ap    = lift2((p: A, q: (a: A) => B) => q(p), p, q);
\end{minted}

With these combinators, it is now possible to parse combinations of primitives, given a completely fixed structure.

\subsection{Alternative}
\label{ssec:alternative}
Note that \texttt{Alternative} exists as a subclass of \texttt{Applicative}, additionally it is closely related to \texttt{MonadPlus}, where the zero elements are \texttt{empty} and \texttt{mzero} respectively, and the combining functions are \texttt{(<|>)} and \texttt{mplus}.
For simplicity, only the \texttt{Alternative} definitions are included.

The \texttt{empty} parser has been previously mentioned in \autoref{ssec:primitives}, as a primitive parser.
Note that combining any parser with the \texttt{empty} parser by using the combining function will result in the original parser.
This has the signature:

\begin{minted}{haskell}
(<|>) :: Parser a -> Parser a -> Parser a
\end{minted}

This combinator captures the ability to choose between alternatives.
For example, the parser \texttt{x <|> y} will first attempt to parse \texttt{x}, however if it fails, it will then attempt to parse \texttt{y}.
Note that with the \textit{Parsec} \cite{leijen01} family of parsers, the second parser is only attempted when the first parser does not consume input.
However, this behaviour may not be desired thus a parser may want to roll back any consumption, or `backtrack'.
Backtracking is done with \texttt{try} (in the case of \textit{Parsec}), where the parser acts as if it did not consume input on a failure.
If both fail, the overall parser fails, however if either succeed, the first to succeed is the result of the parser.
It is crucial to note that while \texttt{\choice} is associative, it is not commutative.

Alternatives, including \texttt{empty}, provide a number of laws that can be beneficial for performing optimisations.
Note that these laws are not always general, but hold for parsers.
These include the left and right neutral laws $(1, 2)$, as well as the left absorption law $(3)$ and left catch law $(4)$:
\begin{align*}
    \texttt{empty \choice p} & \Rightarrow \texttt{p} & (1) \\
    \texttt{p \choice empty} & \Rightarrow \texttt{p} & (2) \\
    \texttt{empty \ap p} & \Rightarrow \texttt{empty} & (3) \\
    \texttt{pure x \choice p} & \Rightarrow \texttt{pure x} & (4)
\end{align*}

The left absorption (zero) law is generally accepted, whereas the right absorption law is not.
In parsers, the parser \texttt{p} in \texttt{p \ap empty} may consume input, thus changing the state.
The left catch law essentially states that \texttt{pure} (from \texttt{Applicative}s) represents a successful parse (specifically in the context of parsers).

\bigskip

By using the \texttt{Alternative} typeclass, the parsers are able to capture choice.
Additionally, this allows for further combinators such as \texttt{many} to be defined, where sequences of arbitrary length can be generated; without choice, a production would either be a fixed finite number, or an infinite length.
For example:

\begin{lstlisting}
A ::= 'a' A -- infinitely many 'a's
B ::= 'b' C -- exactly 2 'b's
C ::= 'b'   -- exactly 1 'b'
\end{lstlisting}

However, with choice, it is therefore possible to define the following, where as many \texttt{'a'}s are parsed as possible, until there are no more \texttt{'a'}s to consume (thus terminating the first branch of the parser (\texttt{'a' A})):

\begin{lstlisting}
A ::= 'a' A | $\varepsilon$
\end{lstlisting}

\subsection{Monad}
\label{ssec:monad}

By using the \texttt{Monad} typeclass, the bind operator is introduced.
This has the signature:

\begin{minted}{haskell}
(>>=) :: Parser a -> (a -> Parser b) -> Parser b
\end{minted}

An interpretation of this combinator is similar to that of \texttt{<*>} in \autoref{ssec:applicative}, however the result of the first parser is applied to a function that creates the second parser, rather than being applied to the result of the second parser (which is a function).
However, as the result of the first parser can directly influence the result of the second parser (since the function can use the result of the first parse), this creates a guarantee on the order of operations.
The first parser must be executed before the function that produces the second parser can be.

While this permits for context-sensitive parsing, as the function producing the second parser can access the results of the previous parse, it severely impacts the efficacy of inspection and analysis.
Due to this decrease in efficacy, bind will not be a primary focus for subsequent optimisations, but it is still important to note due to its power (it can implement both \texttt{\ap} and \texttt{\fmap}).
