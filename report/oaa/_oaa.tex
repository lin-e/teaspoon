\chapter{Optimisation and Analysis}
\label{chap:oaa}

The optimisation stage, introduced in \autoref{chap:preprocessor}, forms the crux of the preprocessor.
This chapter provides a deeper insight into each of the various optimisations and how they can remedy common pitfalls present in parser combinators.
These optimisations include simple changes to the parsers via parser laws to simplify the combinator tree, as well as deeper optimisations which can alter the semantics such as rewriting a grammar to support left recursion.
Multiple optimisations that aid in improving performance from backtracking are also explored, including a specific case for alternative strings as well as a more generalised case on combinators.

\section{Simple Optimisations}
\label{sec:simple_opt}

\subsection{Choice Reduction}
\label{ssec:choice_reduce}

A common pattern in parsers and grammars are sequences of alternatives, or disjunctions.
Since choice (\texttt{\choice}) is an associative operator, it can often be useful to treat a tree of alternatives as a sequence, which can be reconstructed in a fully left-associative manner.
Note that the order should generally be maintained, as it is not a commutative operator.

While this property is primarily used in subsequent optimisations, it can enable a simple optimisation - removing duplicate disjunctions.
For example, if a parser was \texttt{p \choice q \choice p}, it could be simply reduced to \texttt{p \choice q}.
However, by trimming down disjunctions, it reduces the amount of processing required in subsequent steps, most of which rely on dealing with sequences of disjunctions.

\subsection{Parser Law Optimisation}
\label{ssec:parser_law}

In order to simplify the parsers further, before more complex optimisations are applied, basic destructive transformations can be performed.
Note that this step does not apply all optimisation rules, even if it may allow for further optimisation - it only makes changes that will reduce the size of the parser.
Consider the size of the parser as the number of nodes in the combinator tree.
Generally, this refers to the neutral, catch, and absorption laws of alternatives, applicatives, and monoidals (previously discussed in \autoref{ssec:alternative}).
However, this step also applies the definition of `fmap' when applicable - by performing this reduction, it allows for subsequent preprocessing stages to easily determine whether something is a parser or not.

\section{Left Recursion Analysis}
\label{sec:lrec_analysis}

A common pitfall of recursive descent parsers, and by extension parser combinators, is the inability to handle left-recursion.
While techniques exist to manually manipulate a grammar, they can often be mechanical and slow.
Additionally, parser combinators typically contain significantly more information than just the parsing structure, often including semantics on how subtrees (of parsers) should be combined.
This section introduces how left recursion can automatically be detected and rewritten in a form that can terminate via iteration \cite{willis21}, while preserving the desired, user-defined semantics.

\subsection{First Sets}
\label{ssec:first_sets}

In order to begin analysing left recursion, the first step is to determine a programmatic method of checking if a production is left-recursive in the first place.
More importantly, the alternatives (if any) that are not left-recursive also need to be deduced.

The first set of a production can be intuitively thought of as anything that can begin the derivation for a particular rule.
In a traditional grammar, this is simply a union over the first sets of all disjunctions (alternatives) of a rule.
The first set of a rule without disjunctions is the first set of the first element in the rule, which can then be recursively computed.
Finally, the first set of a terminal is a set containing only itself.
The same intuition can be carried forwards to parser combinators.

Let $\mathcal{F}$ represent the \textbf{global} first set and $\mathcal{L}$ represent the \textbf{local} first set.
The local first set is defined as follows on combinators (note that $\texttt{i}_\texttt{0}$ denotes the first character of the string literal that \texttt{i} represents (which can be an escape sequence) and \texttt{<X>} denotes any arbitrary combinator):
\begin{align*}
    \mathcal{L}(\texttt{p \choice q}) & = \mathcal{L}(\texttt{p}) \cup \mathcal{L}(\texttt{q}) & \text{union for choice} \\
    \mathcal{L}(\texttt{f \constfmapl p})\ |\ \mathcal{L}(\texttt{f \fmap p}) & = \mathcal{L}(\texttt{p}) & \text{non-parser on left} \\
    \mathcal{L}(\texttt{p \constfmapr f})\ |\ \mathcal{L}(\texttt{p \pamf f}) & = \mathcal{L}(\texttt{p}) & \text{non-parser on right} \\
    \mathcal{L}(\texttt{pure x <X> p}) & = \mathcal{L}(\texttt{p}) & \text{\texttt{pure} ($\varepsilon$) first} \\
    \mathcal{L}(\texttt{p <X> q}) & = \begin{cases}
        \mathcal{L}(p) & \text{if } \mathcal{L}(p) \neq \varnothing \\
        \mathcal{L}(q) & \text{otherwise}
    \end{cases} & \text{any other combinator} \\
    \mathcal{L}(\texttt{pure x})\ |\ \mathcal{L}(\texttt{empty}) & = \varnothing & \text{empty set} \\
    \mathcal{L}(\texttt{i: Id}) & = \{ \texttt{i} \} & \text{`terminal'}\\
    \mathcal{L}(\texttt{a: Atom}) & = \{ \texttt{a} \} & \text{`terminal'} \\
    \mathcal{L}(\texttt{i: IRLiteral}) & = \begin{cases}
        \{ \texttt{i}_\texttt{0}, \texttt{i} \} & \text{if string / char} \\
        \{ \texttt{i} \} & \text{otherwise}
    \end{cases} & \text{`terminal'}
\end{align*}

The computation of the local first set is done via the use of pattern matching in Scala, with recursion when required.
Also note that the results of the first set computation is memoised in the shared mutable state, preventing a possibly expensive recomputation at the cost of space.

Note that the computation for the first set of a function is similar, by looking at the first parser in the function's arguments.
One important distinction is that the local first set considers identifiers, which are typically references to other parsers (productions), as a terminal, which is incorrect.
While this alone allows for some simple left-recursive productions to be detected, it does not yet fully account for indirect left recursion.
As the preprocessor has a wider view of the program, it is feasible to obtain the global first set of all productions.

Consider all lexical declarations that can occur in the program.
Looking at only parsers (other declarations will have a first set computed, but are meaningless), a declaration refers to any statement that declares a value (reassignments are not supported), such as \texttt{p = q}.
In this case, \texttt{p} is the parser of interest, which is an identifier assignment.
The value of the assignment, \texttt{q}, can be in any of the forms listed above, including disjunctions.

A relation $L$, where $\langle p, a \rangle$ denotes $a$ (a parser) as being in the \textbf{local} first set of $p$ (an identifier), is defined as the following:

$$L = \{ \langle p, a \rangle\ |\ a \in \mathcal{L}(p) \}$$

The global first set can then be computed as the \emph{transitive closure} of $L$, such that $\mathcal{F} = L^+$.
Note that the transitive closure $L^+$ refers to the smallest transitive set containing $L$.
A binary relation $R$ on \texttt{IR} is defined as transitive if for all $p, q, r \in \texttt{IR}$ if $\langle p, q \rangle$ and $\langle q, r \rangle$ both exist, then $\langle p, r \rangle$ must exist, or formally:

$$\forall p, q, r \in \texttt{IR}\ [\langle p, q \rangle \in R \land \langle q, r \rangle \in R \Rightarrow \langle p, r \rangle \in R]$$

This can be computed by adding `missing' tuples by repeatedly performing relation composition until the relation is transitive.
As each step of adding missing tuples is required for transitivity, this creates the minimal set - the transitive closure.
In practice, this computation is done by first scanning the entire program's AST for declarations and computing the local first set, creating a mapping from \texttt{String} (the identifier) to \texttt{Set[IR]}.
The transitive closure is then computed by iteratively expanding any identifiers found in a first set to the identifier's respective first set until all identifiers (in the expanded first set) have been visited.

Consider the following example of a parser that contains indirect left recursion:

\begin{minted}{teaspoon_lex.py:TeaspoonLexer -x}
    val a  = '1' $> 1;
    lazy p = f <$> q <*> a <|> a;
    lazy q = p <* '+';
\end{minted}

The following result is obtained, using the rules for \textbf{local} first sets and the subsequent construction of $\mathcal{F}$.
Note that expansion is an iterative process, as seen in $\mathcal{F}(\texttt{q})$, where the expansion of \texttt{p} leads to further expansion on \texttt{a}.
\begin{align*}
    \mathcal{L}(\texttt{a}) & = \{ \texttt{'1'} \} \\
    \mathcal{L}(\texttt{p}) & = \{ \texttt{q}, \texttt{a} \} \\
    \mathcal{L}(\texttt{q}) & = \{ \texttt{p} \} \\
    \mathcal{F}(\texttt{a}) & = \{ \texttt{'1'} \} & \text{no identifiers} \\
    \mathcal{F}(\texttt{p}) & = \{ \texttt{p}, \texttt{q}, \texttt{a}, \texttt{'1'} \} & \texttt{q} \mapsto \{ \texttt{q}, \texttt{p} \}, \texttt{a} \mapsto \{ \texttt{a}, \texttt{'1'} \} \\
    \mathcal{F}(\texttt{q}) & = \{ \texttt{p}, \texttt{q}, \texttt{a}, \texttt{'1'} \} & \texttt{p} \mapsto \{ \texttt{p}, \texttt{q}, \texttt{a} \} \mapsto \{ \texttt{p}, \texttt{q}, \texttt{a}, \texttt{'1'} \}
\end{align*}

\subsection{Detection}
\label{ssec:lrec_detection}

By computing the first sets $\mathcal{F}$, it is now possible to detect all three forms of left recursion; direct (from the local first set), indirect (via the transitive closure on local first sets), and hidden (via the rules in place to deal with $\varepsilon$).
Recall that a parser is left-recursive when it is able to derive some form, after some number of substitutions (or none), with itself as the first parser \cite{power99}.
Intuitively, this means left recursion occurs when a parser contains a rule where the derivation begins with itself.
As such, a parser \texttt{p} is left-recursive if it is contained within its own first set; $\texttt{p} \in \mathcal{F}(\texttt{p})$.

However, simply detecting a parser is left-recursive is not enough to begin rewriting it.
The more common form is as follows, where some rules ($\texttt{r}_\texttt{i}$) are left-recursive, but others ($\texttt{q}_\texttt{j}$) are not:

\begin{capminted}
    \begin{minted}{teaspoon_lex.py:TeaspoonLexer -x}
        lazy p = r_1 <|> ... <|> r_m <|> q_1 <|> ... <|> q_n;
    \end{minted}
    \vspace{-0.5\baselineskip}
    \caption{Example of parser \texttt{p} in a general form}
    \label{lst:p_general}
\end{capminted}

In order to rewrite the structure of \texttt{p}, the disjunctions must first be partitioned into those which are locally left-recursive (\texttt{R}) and those which are not (\texttt{Q}).
Note that the use of $\text{disjunctions}(\texttt{p})$ refers to the set of alternatives for \texttt{p}.
\begin{align*}
    \text{disjunctions}(\texttt{p}) & = \{ \texttt{r}_\texttt{1}, \dots, \texttt{r}_\texttt{m}, \texttt{q}_\texttt{1}, \dots, \texttt{q}_\texttt{n} \} & \text{from example above} \\
    \texttt{R} & = \{ \pi \in \text{disjunctions}(\texttt{p})\ |\ p \in \mathcal{L}(\pi) \} \\
    \texttt{Q} & = \{ \pi \in \text{disjunctions}(\texttt{p})\ |\ p \notin \mathcal{L}(\pi) \}
\end{align*}

The use of local left recursion (rather than global) allows for safer restructuring - hoisting in declarations may lead to unintended consequences as well as generally more verbose, thus less legible, generated code.
In order to allow for declarations that can be safely hoisted in, the lexical declaration \texttt{inline} has been implemented, as mentioned in \autoref{ssec:inline}.
As such, all three forms of left recursion can be detected, however, only direct left recursion can be rewritten.
The other forms are quite infrequent in most grammars, as mentioned in Parr et al. (2014) \cite{parr14}.

\subsection{Rewriting}
\label{ssec:lrec_rewrite}
It is important to note that by rewriting the productions, the preprocessor will fundamentally alter the semantics of these parsers.
Parsers which previously would not terminate due to infinite recursion are now modified to parse iteratively, with termination being possible.

This is done in three steps; normalisation, reduction, and finally, rewriting.
The underlying idea behind rewriting left recursion is to convert a `standard' production into a parser that utilises \texttt{postfix}.
Note that \texttt{postfix} was chosen (rather than \texttt{infixl}, or similar \cite{willis21}) as it provided the most `general' form - it is able to automatically handle infix operators as well as postfix operators.
This property is desirable, especially in the context of automatic rewriting; while the generated program may be less concise, it allows for more forms of detection, as well as a reduced likelihood of errors, thus requiring less manual intervention.

Consider the following (direct) left-recursive production, similar to the previous declaration of \texttt{p};
$$P \to \underbrace{\overbrace{Ps_1}^{r_1}\ |\ \dots\ |\ \overbrace{Ps_m}^{r_m}}_{\texttt{R}}\ |\ \underbrace{q_1\ |\ \dots\ |\ q_n}_{\texttt{Q}}$$

On a translation to a PEG, the recursive productions simply become repetition operators as stated in Ford (2004) \cite{ford04}, thus the following result is obtained (CFG to PEG);
$$P \leftarrow (q_1\ /\ \dots\ /\ q_n)(s_1\ /\ \dots\ /\ s_m)*$$

This result is similar to that of Hill (1994) \cite{hill94}.
Note that $P$ may appear in $s_j$, such as in the case of addition ($E \to E \texttt{ '+' } E$) where $P = E$ and $s_j = \texttt{'+' } E$.
However, the rewriting remains valid as long as $P \notin \mathcal{F}(s_j)$.
Note that the rewrite still occurs if $P \in \mathcal{F}(s_j)$ but $P \notin \mathcal{L}(s_j)$, however, the preprocessor will raise a warning regarding a possible indirect or hidden left recursion.

However, this translation of a CFG to a PEG must be mirrored in terms of parser combinators, which not only carries syntactic information in terms of the parse structure, but has underlying semantics in terms of how the parsed results are used (or not used).
This raises a number of challenges: the same grammar can exist in multiple forms (lack of normalisation) and combinators carry semantic information as well as syntactic information.

The first of which is that parser combinators can exist in a number of forms which represent the same underlying grammar.
For example, the four following combinators all represent $P_i \to P_i S\ |\ Q$:

\begin{minted}{teaspoon_lex.py:TeaspoonLexer -x}
    lazy p_0 = f_0 <$> p_0 <*> s <|> q;
    lazy p_1 = p_1 <**> pure(f_1) <*> s <|> q;
    lazy p_2 = f_2 <$> p_2 <~> s <|> q;
    lazy p_3 = lift2(f_3, p_3, s) <|> q;
\end{minted}

The second problem is preserving the semantics of the parse - note that in the example above, there is an additional transformation that is applied to each of the $P_i S$ disjunctions.
While it may be quite intuitive to reason about the behaviour of the function at a glance, any changes to the parser, which may be numerous, must be accurately reflected in how the function is transformed (recall the rewriting of certain functions from \autoref{ssec:func_rewrite}).
Without this constraint, a rewritten parser would simply verify if the structure is correct and give the sequence of parsers applied - this is only sufficient for recognisers.
While parser generators will still require tree reassociation, the semantics are provided `externally' (outside the parser generator) over the parse tree, which is not the case for parser combinators.

\subsubsection*{Normalisation}
The first problem is addressed via normalisation and reduction.
In order to aid the subsequent steps, any locally left-recursive parsers are matched against a series of patterns in order to extract the required elements.
The chosen normal form for the majority of operators is $\texttt{p} = \texttt{lift2(f, q, r)}$ - this allows for a clear separation between three key components; \texttt{f} (an uncurried function), \texttt{q} (the left recursion), and \texttt{r} (the `remainder' or suffix).
Notice that the left recursion does not necessarily have to be \texttt{p} - just that $\texttt{p} \in \mathcal{L}(\texttt{q})$.

Note that any patterns using `reverse fmap' (\texttt{\pamf}) will be omitted for brevity; simply replace a use of \texttt{f \fmap p} with \texttt{p \pamf f}, additionally patterns using `const fmap' such as \texttt{x \constfmapl p} can be substituted with \texttt{p \constfmapr x}.
Some normalisation patterns are as follows:

\begin{capminted}
    \begin{minted}{scala}
        p match {
            case (f <$> q) <*> r          => Lift2(Uncurry(f), q, r)
            case (f <$> (q <~> r))        => Lift2(ArgsToTuple(f), q, r)
            case (q <**> Pure(f)) <*> r   => Lift2(Uncurry(f), q, r)
            case q <**> (f <$> r)         => Lift2(Uncurry(Flip(f)), q, r)
            case (q <**> (f <$ op)) <*> r => Lift2(Uncurry(f), q, op *> r)
            // ...
        }
    \end{minted}
    \vspace{-0.5\baselineskip}
    \caption{Examples of normalisation steps for common patterns}
    \label{lst:scala_normalisation}
\end{capminted}

While these patterns cannot account for every possible combination, when coupled with the later reduction steps, they can deal with a variety of parsers in various common forms.
The validity of these normalisation steps is detailed in \autoref{sec:sound_normalisation}.

\subsubsection*{Reduction}
Recall that the recursive component of $\texttt{p} = \texttt{lift2(...)}$ does not necessarily have to be \texttt{p}.
However, the end goal is to `reduce' this component down to just be \texttt{p}.
This is done via a sequence of reduction steps, which monotonically simplifies the recursive component; the complexity of a component can be quantified as the size of the combinator tree representing it.
By requiring a constraint where the structure is monotonically simplified, termination is guaranteed - resulting in a successful reduction or a failed reduction (which causes the preprocessor to roll back any transformations made on a particular disjunction).

Similar to the normalisation step, some patterns will be omitted for brevity if they are equivalent to another pattern, albeit with flipped arguments.
Additionally, \texttt{\apl} and \texttt{\multl} are equivalent and thus have the same reduction rules.
Function shorthand will also contain \texttt{UFA(f)} and \texttt{MFA(f)} to represent \texttt{UnpairFirstArg(f)} and \texttt{MapFirstArg(f)}, respectively.

\begin{capminted}
    \begin{minted}{scala}
        p match {
            case Lift2(f, r <* s, q)   => _Lift2(f, r, s *> q)
            case Lift2(f, g <$> r, q)  => _Lift2(MFA(g, f), r, q)
            case Lift2(f, c <$ r, q)   => _Lift2(MFA(Const(c), f), r, q)
            case Lift2(f, r <~> s, q)  => _Lift2(UFA(f), r, s <~> q)
            case Lift2(f, r <*> s, q)  => _Lift2(f, AFM <$> (r <~> s), q) // (1)
            case Lift2(f, r <**> s, q) => _Lift2(f, PFM <$> (r <~> s), q) // (1)
            case Lift2(f, r <:> s, q)  => _Lift2(f, ((:) <$> r) <*> s, q) // (2)
            // ...
        }
    \end{minted}
    \vspace{-0.5\baselineskip}
    \caption{Some \texttt{lift2} reduction cases, note that \texttt{\_Lift2} is a `smart constructor' that performs further reductions}
    \label{lst:scala_reduction}
\end{capminted}

The validity of these reductions is proven via equivalences in \autoref{sec:sound_reductions}.
Note that the rules for \texttt{\ap} and \texttt{\pa} bootstrap off of the existing reduction steps for \texttt{\mult} and \texttt{\fmap}, based on the idea of recovering applicative from monoidal, as stated by McBride \& Paterson (2008) \cite{mcbride08}.
As such, the reduction is implemented in a way that adds complexity to begin with but reduces in a fixed, finite number of steps to a simpler form, thus maintaining the monotonically decreasing constraint.
The same idea is applied for \texttt{<:>} (lifted cons), however this also builds on the use of \texttt{\ap}.

\subsubsection*{Postfix}

Prior to this point, the parsers were still left-recursive.
Nothing has been done to change the semantics yet, however left-recursive disjunctions have been `reconditioned'; the aforementioned problems (the lack of a normal form and the preservation of parse semantics) have been dealt with.
Recall the earlier example of \texttt{p}, shown in \autoref{lst:p_general}.
In the ideal case, this parser would have become reconditioned to the following;
\begin{capminted}
    \begin{minted}{teaspoon_lex.py:TeaspoonLexer -x}
        lazy p = lift2(f_1, p, s_1) <|> ... <|> lift2(f_m, p, s_m) <|>
                 q_1 <|> ... <|> q_n;
    \end{minted}
    \vspace{-0.5\baselineskip}
    \caption{Example of parser \texttt{p}, after normalisation and reduction}
    \label{lst:p_norm_reduce}
\end{capminted}

Using the previous partitioning of \texttt{Q} and \texttt{R}, where the latter represents left-recursive disjunctions, it follows that the newly created \texttt{lift2}s fall into the \texttt{R} partition.
Note that \texttt{postfix} and \texttt{lift2} have the following types - additionally, in the case of \texttt{lift2} there is the constraint that $\texttt{C} = \texttt{A}$, as \texttt{p} has the type \texttt{P<A>}.

\begin{capminted}
    \begin{minted}{typescript}
        function postfix(q: P<A>, op: P<(a: A) => A>): P<A>;
        function lift2(f: (a: A, b: B) => C, p: P<A>, q: P<B>): P<C>;
    \end{minted}
    \vspace{-0.5\baselineskip}
    \caption{Types involved for postfix conversion}
    \label{lst:conv_types}
\end{capminted}

Using these types as guide, the next natural step is to populate the parameters of \texttt{postfix} as required, resulting in a parser that replaces recursion with iteration.
The semantics of \texttt{postfix} are that it parses one \texttt{q} and then zero or more occurrences of \texttt{op}, applying the result to some accumulating value.
Naturally, \texttt{q} must be the non-recursive cases found in the \texttt{Q} partition.
However, \texttt{Q} represents a collection of parsers, which can trivially be changed into a parser by reducing as alternatives (\texttt{\choice}).

The remaining work lies in converting the recursive cases into \texttt{op}s (which can then be reduced in the same way).
Recall that work was done to ensure that \texttt{lift2} followed the same structure, where \texttt{p} is isolated in all alternatives.
Consider a single recursive disjunction, $\texttt{lift2(f, p, s)} \in \texttt{R}$.
If the value for \texttt{b} (the second argument) were to be pre-populated in \texttt{f}, the resulting function would be ideal for \texttt{postfix}.

This can be seen in \autoref{lst:sub1_p}, where the parser simply subtracts 1 from the number parsed.
In this example, the argument \texttt{b} is represented by \texttt{y}.
However, it is clear that this will be 1: the result of the second `parser' (\texttt{pure(1)}).
If this value were to be populated, the resultant function would resemble \texttt{(x: number) => x - 1}, which is the desired effect.
Of course, this would have to be done to accommodate an arbitrary parsed result, not just a constant.

\begin{capminted}
    \begin{minted}{typescript}
        let sub1 = lift2((x: number, y: number) => x - y, nat, pure(1));
    \end{minted}
    \vspace{-0.5\baselineskip}
    \caption{Simple example of \texttt{lift2} to demonstrate pre-populating an argument}
    \label{lst:sub1_p}
\end{capminted}

This is done by first currying \texttt{f}, which has the type \texttt{(a: A) => (b: B) => A}.
By flipping this curried function, the resultant type is \texttt{(b: B) => (a: A) => A}.
Finally, this can be partially applied with `fmap' (\texttt{\fmap}), with the result (\texttt{flip(curry(f)) \fmap r}) being of the desired type.
This transformation is performed on all alternatives in \texttt{R}, which is then reduced.
Validity of this transformation is further discussed in \autoref{sec:sound_rewrite}.

The earlier example, shown in \autoref{lst:p_general} and \autoref{lst:p_norm_reduce}, is finally rewritten into:
\begin{capminted}
    \begin{minted}{teaspoon_lex.py:TeaspoonLexer -x}
        lazy p = postfix(q_1 <|> ... <|> q_n,
            flip(curry(f_1)) <$> s_1 <|> ... <|>
            flip(curry(f_m)) <$> s_m
        );
    \end{minted}
    \vspace{-0.5\baselineskip}
    \caption{Example of parser \texttt{p}, after rewriting}
    \label{lst:p_rewrite}
    \vspace{-\baselineskip}
\end{capminted}

\subsection{Worked Example}
\label{ssec:worked_ex}

Recall the example of addition first introduced in \autoref{lst:running_example} and the subsequent IR in \autoref{lst:running_ir}.
The code matches the pattern \texttt{q \pa (f \constfmapl op) \ap r} exactly, with no requirement for reduction.
In this case (let \texttt{C('+')} denote \texttt{Chr(IRLiteral("'+'"))} for brevity):

\begin{center}
    \vspace{-2\baselineskip}
    \begin{minipage}[t]{0.49\textwidth}
        \begin{align*}
            \texttt{q} & = \texttt{Id("expr")} \\
            \texttt{f} & = \texttt{Atom(0)}
        \end{align*}
    \end{minipage}
    \hfill
    \begin{minipage}[t]{0.49\textwidth}
        \begin{align*}
            \texttt{op} & = \texttt{C('+')} \\
            \texttt{r} & = \texttt{Id("nat")}
        \end{align*}
    \end{minipage}
\end{center}

Trivially, this is converted into \texttt{lift2(u(f), Id("expr"), C('+') \apr Id("nat"))}.
As this is already in the desired shape, where the left recursion is the first parser, it can be converted into a \texttt{postfix} operation.
This is then rewritten into the following (the \violet{violet} component is the result of the function after converting from the IR):

$$\underbrace{\texttt{flip(curry(u(f)))}}_{\violet{\texttt{(y) => (x) => x + y}}} \texttt{ \fmap (C('+') \apr Id("nat"))}$$

Notice how the first application, which would be to the result of \texttt{Id("nat")}, is the argument \texttt{y}, thus pre-populating the function as desired.
\section{String Trie}
\label{sec:str_trie}

A frequently observed pattern in parsing, especially in programming languages, is a number of alternative strings, namely keywords.
However, this pattern can often lead to two pitfalls; the first revolves around the ordering of words and the second revolves around the performance penalty caused by backtracking.
The execution of disjunctions in parser combinators follows the behaviour of PEGs.
As such, the `dangling else ambiguity' can be easily resolved by a reordering of disjunctions, with the longest pattern first.
This also means that disjunctions are not commutative (therefore, this optimisation changes the semantics of the parser).
Consider an example consisting of a subset of TypeScript keywords:

\begin{lstlisting}
    kw ::= 'as' | 'async' | 'break' | 'case' | 'const' | 'continue'
\end{lstlisting}

In this order, if the word \texttt{'async'} were to be parsed, the parser would terminate on a successful parse of \texttt{'as'}, with a remainder of \texttt{'ync'}.
On the other hand, if the order were to be flipped, with \texttt{'as'} being after \texttt{'async'}, the parser would fail due to the input being partially consumed - backtracking is therefore required to reset the state back to where the first parser began.
However, backtracking operations are inherently expensive, possibly requiring work to be redone.
Ideally, a parser would have minimal (if any) backtracking (implemented via the use of \texttt{attempt}), and any parsers that contain backtracking would be as `small' as possible.

As noted in Swierstra (2000) \cite{swierstra00}, constructing a `trie' structure allows for possible productions to be grouped by common prefixes, thus eliminating the need for backtracking - ambiguities are resolved as they would be separated into choices that have no conflict.
Without any backtracking, the parser is able to complete in linear time, after this data structure has been constructed.
Rather than having this pattern detection and construction being performed at runtime (in TypeScript), this can be done during transpile time (in Scala, within the preprocessor).

While this approach can limit the effectiveness due to its inability to inspect values only known at runtime, most common patterns (such as the motivating example) leverage string literals as keywords, which can trivially be accessed and analysed by the preprocessor.
By moving the cost of the analysis to the preprocessor, it removes the performance penalty of constructing the trie on the initial run of the parser, which would likely cause an overall performance degradation if the parser is not commonly used or is sufficiently small.

\begin{figure}[H]
    \centering
    \begin{tikzpicture}[every node/.style={execute at end node=\vphantom{bg}}]
        \begin{scope}[shift={(0, 0)}]
            \node[anchor=west] (o) at (0, 0) {\texttt{?}};
            \node[anchor=west] (a) at (1, 1) {\texttt{a}};
            \node[anchor=west] (b) at (1, 0) {\texttt{b}};
            \node[anchor=west] (c) at (1, -1) {\texttt{c}};
            \node[anchor=west, draw] (as) at (2, 1) {\texttt{s}};
            \node[anchor=west] (asy) at (3, 1) {\texttt{y}};
            \node[anchor=west] (asyn) at (4, 1) {\texttt{n}};
            \node[anchor=west, draw] (async) at (5, 1) {\texttt{c}};
            \node[anchor=west] (br) at (2, 0) {\texttt{r}};
            \node[anchor=west] (bre) at (3, 0) {\texttt{e}};
            \node[anchor=west] (brea) at (4, 0) {\texttt{a}};
            \node[anchor=west, draw] (break) at (5, 0) {\texttt{k}};
            \node[anchor=west] (ca) at (2, -1) {\texttt{a}};
            \node[anchor=west] (cas) at (3, -1) {\texttt{s}};
            \node[anchor=west, draw] (case) at (4, -1) {\texttt{e}};
            \node[anchor=west] (co) at (2, -2) {\texttt{o}};
            \node[anchor=west] (con) at (3, -2) {\texttt{n}};
            \node[anchor=west] (cons) at (4, -2) {\texttt{s}};
            \node[anchor=west, draw] (const) at (5, -2) {\texttt{t}};
            \node[anchor=west] (cont) at (4, -3) {\texttt{t}};
            \node[anchor=west] (conti) at (5, -3) {\texttt{i}};
            \node[anchor=west] (contin) at (6, -3) {\texttt{n}};
            \node[anchor=west] (continu) at (7, -3) {\texttt{u}};
            \node[anchor=west, draw] (continue) at (8, -3) {\texttt{e}};

            \draw
            (o) -- (a) -- (as) -- (asy) -- (asyn) -- (async)
            (o) -- (b) -- (br) -- (bre) -- (brea) -- (break)
            (o) -- (c)
                   (c) -- (ca) -- (cas) -- (case)
                   (c) -- (co) -- (con)
                                  (con) -- (cons) -- (const)
                                  (con) -- (cont) -- (conti) -- (contin) -- (continu) -- (continue);
        \end{scope}
        \node[anchor=west] at (7, 0) {$\leadsto$};
        \begin{scope}[shift={(9, 0)}]
            \node[anchor=west] (o) at (0, 0) {\texttt{?}};
            \node[anchor=west] (as) at (1.5, 1) {\texttt{as}};
            \node[anchor=west, draw] (asE) at (3, 2) {\texttt{\_}};
            \node[anchor=west, draw] (async) at (3, 1) {\texttt{ync}};
            \node[anchor=west, draw] (break) at (1.5, 0) {\texttt{break}};
            \node[anchor=west] (c) at (1.5, -1) {\texttt{c}};
            \node[anchor=west, draw] (case) at (3, -1) {\texttt{ase}};
            \node[anchor=west] (con) at (3, -2) {\texttt{on}};
            \node[anchor=west, draw] (const) at (4.5, -2) {\texttt{st}};
            \node[anchor=west, draw] (continue) at (4.5, -3) {\texttt{tinue}};
            \draw
            (o) -- (as)
                   (as) -- (asE)
                   (as) -- (async)
            (o) -- (break)
            (o) -- (c)
                   (c) -- (case)
                   (c) -- (con)
                          (con) -- (const)
                          (con) -- (continue);
        \end{scope}
    \end{tikzpicture}
    \caption{
        Left: uncompressed (na\"{i}ve) trie, right: compressed (radix) trie.
        The initial \texttt{?} denotes the root, a box denotes a `complete' word, and \texttt{\_} denotes the empty ($\varepsilon$) string, \texttt{pure("")}.
    }
    \label{fig:trie}
\end{figure}

The preprocessor implements this optimisation by first detecting chains of \texttt{Str} or \texttt{Chr} (in the IR) and attempting to extract a list of string or character literals.
These literals are then inserted into an uncompressed trie (one character at a time) in order to allow for escape sequences to be easily handled.
Once all words have been inserted, the trie is compressed, as seen in \autoref{fig:trie}.
This structure is then converted back into a parser bottom-up; the order of alternatives does not matter as long as the empty string (if present) is at the end.
Note that a bottom-up (leaf to root) construction is only valid if the combinators or operations to join the parsers are associative, or are naturally right-associative.
Fortunately, in the case of string concatenation, this holds.
\section{Backtracking Reduction}
\label{sec:backtrack_reduction}

As mentioned in \autoref{sec:str_trie}, there is a significant performance penalty incurred by backtracking.
However, the technique used for strings does not generalise well to arbitrary sequences of combinators due to the lack of right-associativity.
This section explores alternative techniques and methods to extract `common' terms in an attempt to reduce backtracking\footnote{No pun intended}.

\subsection{Flat IR and Na\"ive Fusion}
\label{ssec:flat_ir}
While the current tree-based IR allows for effective analysis on the localised \textbf{structure} of the parse, it can often become cumbersome when reasoning about the \textbf{sequence} of the parse; the order in which parsers are applied (and more importantly, consume input).
This becomes particularly tedious when inspection needs to be done on the whole parser tree, when the leftmost and rightmost leaves need to be inspected.
Due to this limitation, inspection in this stage is done primarily using the \texttt{FlatParser} type, which is a sequence of \texttt{FlatIR}s.
Note that \texttt{FlatIR} is a trait which contains two notable concrete implementations; \texttt{P} and \texttt{C} - the former wraps an \texttt{IR} and the latter represents a \texttt{Combinator}.
The conversion is done respecting left-associativity; therefore traversal is only performed on the left-hand side of a combinator, as seen in \autoref{fig:flat_ir}.

\begin{figure}[H]
    \centering
    \begin{tikzpicture}[every node/.style={execute at end node=\vphantom{bg}}]
        \begin{scope}[shift={(0, 0)}]
            \node (o) at (0, 0) {\texttt{\ap}};
            \node (ol) at (-1, -1) {\texttt{\ap}};
            \node (or) at (1, -1) {\texttt{\mult}};
            \node (oll) at (-1.5, -2) {\texttt{p}};
            \node (olr) at (-0.5, -2) {\texttt{q}};
            \node (orl) at (0.5, -2) {\texttt{r}};
            \node (orr) at (1.5, -2) {\texttt{s}};
            \draw
            (o) -- (ol)
            (o) -- (or)
            (ol) -- (oll)
            (ol) -- (olr)
            (or) -- (orl)
            (or) -- (orr);
        \end{scope}
        \node[anchor=west] at (2, -1) {$\leadsto$};
        \begin{scope}[shift={(3, -1)}]
            \node[anchor=west] at (0, 0) {\texttt{[P(p), C(\ap), P(q), C(\ap), P(r \mult s)]}};
        \end{scope}
    \end{tikzpicture}
    \caption{
        Left: tree-based (original) IR, right: list-based (flat) IR.
        Both represent the parser \texttt{p \ap q \ap (r \mult s)}.
    }
    \label{fig:flat_ir}
\end{figure}

Since left-associativity is respected when converting from the combinator tree to the flat representation, recovery can be done without any concerns of changing associativity.
However, this approach limits the amount of inspection that can be performed, especially when considering common terms.
For example, in \autoref{fig:flat_ir}, inspection cannot be done within \texttt{\mult}, as it is considered a single parser.

Certain combinators, namely the label operation (\texttt{<?>}) and choice (\texttt{\choice}) are not flattened.
Any other nodes in the IR, including functions, are kept as `singleton' parsers.

This representation allows for a na\"ive implementation of factoring.
Two disjunctions can be `fused' when everything, other than the last element in the list, is equal between both parsers.
This element can then be combined by utilising the distributive property of alternatives on applicatives.
Note that this form of fusion is omitted from the pipeline, as another technique discussed later is generally more effective.

Also note that the left-distributive property\footnote{\texttt{a <*> (b <|> c) = (a <*> b) <|> (a <*> c)}} is only applicable for backtracking parsers and parsers with non-biased choice (the \textit{Parsec} family is left-biased).
However, as all disjunctions require backtracking in order to benefit from this optimisation, it can be assumed to hold within this optimisation.
The right-distributive\footnote{\texttt{(a <|> b) <*> c = (a <*> c) <|> (b <*> c)}} property does not hold for parsers \cite{typeclassopedia}, as demonstrated in \autoref{lst:no_right_fact} with an input of \texttt{'+++'}.
In this scenario, \texttt{c} will only be able to succeed if \texttt{b} succeeds (if \texttt{a} succeeds, there will be insufficient \texttt{'+'}s for \texttt{c}) - this case is possible in \texttt{p}, as the first disjunction will fail.
However, if \texttt{q} was to be used, the first conjunction would succeed with \texttt{a} succeeding, however \texttt{c} will then fail, causing the entire parser to fail.
Furthermore, factoring on the right does not offer a performance benefit - the same amount of backtracking would still be performed, however code size may be reduced.

\begin{capminted}
    \begin{minted}{teaspoon_lex.py:TeaspoonLexer -x}
        let a = str('++');
        let b = str('+');
        let c = str('++');
        let p = attempt(a *> c) <|> attempt(b *> c);
        let q = (attempt(a) <|> attempt(b)) *> c;
    \end{minted}
    \vspace{-0.5\baselineskip}
    \caption{Example showing failure of right factoring, \texttt{p} is the original parser and \texttt{q} is the attempted optimisation}
    \label{lst:no_right_fact}
\end{capminted}

\subsection{Left-Associative Normalisation}
The crux of this technique is to `fuse' similar structures with alternative parsers.
In order to maximise the efficacy of this technique, a normal form should be established as it permits parser structures that are semantically equivalent that otherwise wouldn't match based on structure to be fused.

Similar to before, cases for \texttt{\multl} and \texttt{\multr} are interchangeable with \texttt{\apl} and \texttt{\apr}, respectively.
The following associativity and reassociation equivalences are applied in order to obtain a normal form:
\begin{align*}
    \texttt{p \apl (q \apl r)} & \Rightarrow \texttt{p \apl q \apl r} \\
    \texttt{p \apr (q \apr r)} & \Rightarrow \texttt{p \apr q \apr r} \\
    \texttt{p \apr (q \ap r)} & \Rightarrow \texttt{p \apr q \ap r} \\
    \texttt{p \ap (q \apl r)} & \Rightarrow \texttt{p \ap q \apl r}
\end{align*}

While this alone does not guarantee everything is converted into a normal form, it increases the amount of left-associated operators.
% By performing this step, the length of the flattened IR becomes significantly larger, which allows for more optimisation opportunities.
% Additionally, further steps that traverse the structure of the combinator tree are able to progress further without requiring additional logic for matching equivalent structures.
By performing this step, further steps that traverse the structure of the combinator tree are able to progress further without requiring additional logic for matching equivalent structures.

\subsection{Fusion with Merkle Trees}
As mentioned earlier, a key limitation of the flattened structure is the inability to analyse the right subtree of a combinator, when it is not atomic (when the parser within \texttt{P} is actually a combinator).
An example of this can be seen in \autoref{fig:merkle_fusion_example}; which would otherwise be fused to \texttt{(r \mult s)/(t \mult s)}.
Another key limitation is that fusion on the flattened IR requires both to have the same number of elements.

\begin{figure}[H]
    \centering
    \begin{tikzpicture}[every node/.style={execute at end node=\vphantom{bg}}]
        \begin{scope}[shift={(-0.5, 0)}]
            \node (o) at (0, 0) {\texttt{\ap}};
            \node (ol) at (-1, -1) {\texttt{\ap}};
            \node (or) at (1, -1) {\texttt{\mult}};
            \node (oll) at (-1.5, -2) {\texttt{p}};
            \node (olr) at (-0.5, -2) {\texttt{q}};
            \node (orl) at (0.5, -2) {\texttt{r}};
            \node[red] (orr) at (1.5, -2) {\texttt{s}};
            \draw
            (o) -- (ol)
            (o) -- (or)
            (ol) -- (oll)
            (ol) -- (olr)
            (or) -- (orl)
            (or) -- (orr);
        \end{scope}
        \node at (2, -1) {\texttt{\choice}};
        \begin{scope}[shift={(4.5, 0)}]
            \node (o) at (0, 0) {\texttt{\ap}};
            \node (ol) at (-1, -1) {\texttt{\ap}};
            \node (or) at (1, -1) {\texttt{\mult}};
            \node (oll) at (-1.5, -2) {\texttt{p}};
            \node (olr) at (-0.5, -2) {\texttt{q}};
            \node (orl) at (0.5, -2) {\texttt{r}};
            \node[red] (orr) at (1.5, -2) {\texttt{t}};
            \draw
            (o) -- (ol)
            (o) -- (or)
            (ol) -- (oll)
            (ol) -- (olr)
            (or) -- (orl)
            (or) -- (orr);
        \end{scope}
        \node at (7, -1) {$\leadsto$};
        \begin{scope}[shift={(9.5, 0)}]
            \node (o) at (0, 0) {\texttt{\ap}};
            \node (ol) at (-1, -1) {\texttt{\ap}};
            \node (or) at (1, -1) {\texttt{\mult}};
            \node (oll) at (-1.5, -2) {\texttt{p}};
            \node (olr) at (-0.5, -2) {\texttt{q}};
            \node (orl) at (0.5, -2) {\texttt{r}};
            \node[red] (orr) at (1.5, -2) {\texttt{s/t}};
            \draw
            (o) -- (ol)
            (o) -- (or)
            (ol) -- (oll)
            (ol) -- (olr)
            (or) -- (orl)
            (or) -- (orr);
        \end{scope}
    \end{tikzpicture}
    \caption{
        Left: two parser trees before fusion, right: resulting parser tree after fusion.
        Nodes in \red{\texttt{red}} are fused.
    }
    \label{fig:merkle_fusion_example}
\end{figure}

However, the problem of efficiently detecting small differences between two trees and performing some aggregation or synchronisation mirrors that of large-scale data management systems such as \textit{DynamoDB}, as described by DeCandia et al. (2007) \cite{decandia07}.
One of the data structures used for `anti-entropy' is a Merkle tree \cite{merkle87}.

Each node contains a hash which consists of the hashes of its children - for the combinator tree, this also needs to account for the operator itself.
Another modification for traversal on combinator trees is that the termination condition is not only on the leaf but rather when a mismatch is detected that prevents further traversal.

The algorithm consists of two components; `matching', where two trees are checked for compatibility and the wrapper, which performs the fusion across disjunctions.
The following describes the matching step.
Assuming that the nodes represent the same combinator and are not equal, the general rule is to first determine whether the right side of the combinator can be fused.
If the left side is equal, it is traversed further until it finds the first `difference' (the combinators are not equal or only one of the two arguments are combinators), at which point a fusion occurs.
For example, if a parser is factorable on the left and the left sides are equal, then the fusion function is called with the two right-hand sides.
Note that if this is the first call of the function, no fusion is performed.
These rules are detailed further in \autoref{alg:merkle_match}.

\begin{algorithm}[H]
    \begin{algorithmic}[1]
        \Require $\text{\sc Type}(a) = \text{\sc Type}(b)$
        \Function{Match}{$a, b, f$} \Comment{$a$ originates from the candidate set}
            \State \textbf{if} $a = b$ \textbf{return} $\text{\sc Some}(a)$
            \State $\text{\sc sc} \gets \text{\sc SameCombinator}(a, b)$ \Comment{also checks if both are combinators}
            \State \textbf{if} {\sc sc} and {\sc lf} and $a.l = b.l$ \textbf{return} $\text{\sc Match}(a.r, b.r, \bot)$ \textbf{map} $\lambda r \to \text{\sc Op}(a.l, r)$
            \State \textbf{if} $\text{\sc ToFuse}(irs) \gets a$ \textbf{return} $\text{\sc Some}(\text{\sc ToFuse}(b :: irs))$
            \State \textbf{if} $f$ \textbf{return} $\text{\sc None}$
            \State \textbf{return} $\text{\sc Some}(\text{\sc ToFuse}([a, b]))$
        \EndFunction
    \end{algorithmic}
    \caption{Matching over Merkle trees, {\sc lf} denotes left-factorable}
    \label{alg:merkle_match}
\end{algorithm}

The fusion stage wraps the matching stage.
Initially, the set of candidates is initialised to the empty set.
For each disjunction, a match is attempted across each of the candidate trees, with the first match being taken (if any).
If a match is successful, the original tree (from the candidate set) is replaced with the new tree (containing the fusion with the disjunction).
Otherwise, the disjunction is added directly to the candidate set.
The purpose of the candidate set is to maintain existing matches as an accumulator, preventing redundant computation and additional parsers from being created.
In practice, this fusion is recursive (the newly combined parser is further optimised) - however, this is done at the end.
These steps are detailed in \autoref{alg:merkle_fuse}.

\begin{algorithm}[H]
    \begin{algorithmic}[1]
        \Require $\exists T\ [\forall ir \in irs\ [\text{\sc Type}(ir) = T]]$ \Comment{all have same type}
        \Function{FuseTrees}{$irs$} \Comment{primary function}
            \State $candidates \gets \varnothing$
            \ForAll{$ir \in irs$}
                \If{$\exists c \in candidates\ [\text{\sc Some}(f) \gets \text{\sc Match}(c, ir, \top)]$}
                    \State $candidates \gets (candidates - \{ c \}) \cup \{ f \}$
                \Else
                    \State $candidates \gets candidates \cup \{ ir \}$
                \EndIf
            \EndFor
            \State \textbf{return} $candidates$ \textbf{map} $\text{\sc Expand}$ \Comment{recursive expand}
        \EndFunction
    \end{algorithmic}
    \caption{Wrapper function for fusion using Merkle trees.}
    \label{alg:merkle_fuse}
\end{algorithm}

Recall that this analysis occurs over disjunctions on a parser.
This provides a crucial invariant; the types of each disjunction must be the same.
As long as this invariant is maintained, the fusions are type-safe.
For example, consider two parsers, \texttt{p <X> q} and \texttt{p <X> r} - if they are of the same type, then \texttt{q} and \texttt{r} must also have the same type.
A recursive step on \texttt{q} and \texttt{r} will also therefore maintain this invariant.
The same argument is made for the other side of the combinator.

\subsection{Ordering with Tries}
Tries also provide another benefit; they can be used to determine the ideal ordering of parsers to prevent the `dangling else' ambiguity.
This property is used for the `string tries' optimisation discussed in \autoref{sec:str_trie}.
However, in order to generalise this to parsers, the preprocessor must first establish the sequence in which parsers execute within the disjunction.
A sequence can be obtained using the existing flattened IR, however, expansion should also be done on the right-hand side.
Recall that the purpose of only expanding on the left-hand side was to allow for easier reconstruction due to left-associativity - a property that is no longer required as this parse sequence will not be reconstructed.

Again, consider the example presented in \autoref{fig:merkle_fusion_example}.
Suppose no fusion had occurred, thus only looking at the two disjunctions.
The two combinator trees would then be flattened into \texttt{[P(p), C(\ap), P(q), C(\ap), P(r), C(\mult), P(s)]} and \texttt{[P(p), C(\ap), P(q), C(\ap), P(r), C(\mult), P(t)]} respectively.
However, when considering the order in which parsers are applied, the combinators are no longer required and can therefore be stripped out, leaving the sequence to be \texttt{[p, q, r, s]} and \texttt{[p, q, r, t]}, respectively.
Note the process of retaining only parsers in the example above is overly simplified.
In reality, larger portions of the flattened parser are observed at a time, similar to peephole optimisation \cite{mckeeman65}.
This filters out `non-parsers', such as functions when used with \texttt{\fmap}.

While this method of determining the parser sequence allows for an order to be established, it does not allow for parsers to be recovered since the combinators have been discarded.
In order to remedy this, the structure of a trie is modified; rather than marking a node as being complete, it needs to contain a collection of parsers, which is initially empty.
On completion of an insert, when the sequence of parsers is empty, the original parser (before flattening) must be added to the collection of complete parsers in a given node.
A post-order traversal can then be performed on the trie, where the completed collection is appended to the end of the results of its child nodes.
Since each of the child nodes represents different parsers (with ambiguity being accounted for), adding the result at the end allows for the `shortest' parser to be performed last.

\subsection{Overall Process}
The previous sections discussed multiple techniques to deal with an abundance of backtracking as well as poorly ordered disjunctions.
However, the actual implementation relies on all three methods: normalisation, fusion with Merkle trees, and trie ordering.

As a \texttt{PipelineStage}, the entry point of this optimisation exists when processing an \texttt{IR}.
Note that this is only run when it detects a sequence of disjunctions, all of which are wrapped in an \texttt{attempt}, as this permits for backtracking - a property that many of the optimisations rely on for fusion.
Once this is detected and collected as a sequence of disjunctions (with the \texttt{attempt}s stripped), normalisation is performed in order to maximise the efficacy of later analysis.
Next, fusion is performed using the Merkle tree - note that any fusion that occurs in this step is recursive and will undergo the same process.
At this stage, the result is still a collection of disjunctions, hopefully containing fewer elements than the original due to fusion.

The final step is to combine the disjunctions.
However, recall that backtracking was removed for the sake of processing.
If the parsers were na\"ively combined, with no regard to backtracking, there would likely be ambiguities between disjunctions, however, if all parsers were wrapped and combined, the primary purpose of this optimisation would be lost.
`Ambiguous' parsers are first wrapped with an \texttt{attempt} and then fed into the trie to perform an ordering, whereas parsers with no ambiguity are directly inserted into the tree - note that the sequence of parsers for a wrapped parser and unwrapped parser is the same.
Not only does this process account for ambiguity, redundant uses of \texttt{attempt} are also removed.

An ambiguity between two parsers \texttt{p} and \texttt{q} occurs when there is some element that exists in the first set of both parsers (defined in \autoref{ssec:first_sets});
$$\text{ambiguous}(\texttt{p}, \texttt{q}) = \exists r\ [r \in \mathcal{F}(\texttt{p}) \land r \in \mathcal{F}(\texttt{q})]$$

This is a minor simplification - additional logic is in place to prevent two `different' parsers from not being a match.
For example, two \texttt{satisfy} parsers may have two functions that are different but can be satisfied by the same character, thus consuming input.
The same can be said for \texttt{Atom}s or identifiers, which cannot be easily inspected - this further motivates the use of the `inlining' functionality introduced in \autoref{ssec:inline}.
In these cases, where it is non-trivial to inspect, the parsers are marked as ambiguous.

Finally, the trie is traversed, which gives the desired ordering (with ambiguous parsers being wrapped).
This can then be reduced by combining all terms with the choice operator.

\section*{Summary}
Throughout this chapter, a number of techniques for improving the performance or usability of parsers have been explored.
These techniques range from `lawful' optimisations based on destructive parser laws, to automated grammar refactoring which fundamentally changes the semantics of a parser, to optimisations that reduce the amount of backtracking performed.
The efficacy of these optimisation stages is analysed further in \autoref{sec:ev_oaa}.

In the case of grammar refactoring, a number of rules were stated with little to no evidence, aside from intuition.
These equivalences and conversions are explored in greater detail in \autoref{chap:soundness}, where the soundness of the aforementioned transformations are shown.
