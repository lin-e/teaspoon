\section{Simple Optimisations}
\label{sec:simple_opt}

\subsection{Choice Reduction}
\label{ssec:choice_reduce}

A common pattern in parsers and grammars are sequences of alternatives, or disjunctions.
Since choice (\texttt{\choice}) is an associative operator, it can often be useful to treat a tree of alternatives as a sequence, which can be reconstructed in a fully left-associative manner.
Note that the order should generally be maintained, as it is not a commutative operator.

While this property is primarily used in subsequent optimisations, it can enable a simple optimisation - removing duplicate disjunctions.
For example, if a parser was \texttt{p \choice q \choice p}, it could be simply reduced to \texttt{p \choice q}.
However, by trimming down disjunctions, it reduces the amount of processing required in subsequent steps, most of which rely on dealing with sequences of disjunctions.

\subsection{Parser Law Optimisation}
\label{ssec:parser_law}

In order to simplify the parsers further, before more complex optimisations are applied, basic destructive transformations can be performed.
Note that this step does not apply all optimisation rules, even if it may allow for further optimisation - it only makes changes that will reduce the size of the parser.
Consider the size of the parser as the number of nodes in the combinator tree.
Generally, this refers to the neutral, catch, and absorption laws of alternatives, applicatives, and monoidals (previously discussed in \autoref{ssec:alternative}).
However, this step also applies the definition of `fmap' when applicable - by performing this reduction, it allows for subsequent preprocessing stages to easily determine whether something is a parser or not.
