\section{Porting from Parsec}
\label{sec:port}

It is important to first acknowledge that a significant amount of the underlying implementation is built off of the work by Hutton \& Meijer (1999) \cite{hutton99}, as well as the later work on \textit{Parsec} by Leijen \& Meijer (2001) \cite{leijen01}.
However, a number of considerations must be made as code is essentially being ported from an implementation in Haskell to TypeScript.

The internal state contains both the position as well as the remaining string, similar to the implementation in Haskell.
Similarly, the consumer-based approach was also taken, where both the success of the parse and consumption is tracked - closely related to the four tags proposed by Partridge \& Wright (1996) \cite{partridge96}.
In TypeScript, consumption is simply a flag that is set in a \texttt{Result<A>}, which also contains a reply, as shown in \autoref{fig:4tags}.

\begin{figure}[H]
    \centering
    \begin{minipage}{0.525\textwidth}
        \begin{minted}{haskell}
            data Consumed a = Consumed (Reply a)
                            | Empty (Reply a)
            data Reply a    = Ok a State Msg
                            | Err Msg
        \end{minted}
    \end{minipage}
    \hfill
    $\leadsto$
    \hfill
    \begin{minipage}{0.35\textwidth}
        \begin{minted}{typescript}
            type Result<A> = {
                consumed: boolean;
                reply: Ok<A> | Err;
            };
        \end{minted}
    \end{minipage}
    \caption{Left: four `tags' from \textit{Parsec} (Haskell), right: representation of the tags in \textit{Teaspoon} (TypeScript)}
    \label{fig:4tags}
\end{figure}

However, a significant language feature present in Haskell but not TypeScript is the presence of lazy evaluation.
An incomplete implementation of laziness, which does not support sharing (preventing repeated evaluations), can be implemented by a function which takes in a `thunk' (a function which can be evaluated later to delay the computation).
This `thunk' is then only evaluated once the lazy parser is called, thus allowing for values to be used prior to an assignment - a critical feature for recursively defined grammars.

\begin{capminted}
    \begin{minted}{typescript}
        function lazy<A>(p: () => Parser<A>): Parser<A> {
            return (s: State) => p()(s);
        }

        let p = lazy(() => q); // 'Using' q before it is assigned
        let q = pure(true);    // Actual definition of q
    \end{minted}
    \vspace{-0.5\baselineskip}
    \caption{Rudimentary implementation of laziness (delayed evaluation only)}
    \label{lst:old_lazy}
\end{capminted}

While this port creates a useful starting point, the performance leaves much to be desired.
Not only are some features implemented in a way that is not natively `supported' (not as well optimised) in TypeScript, some features benefit from a less `pure' implementation.
