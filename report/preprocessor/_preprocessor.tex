\chapter{Preprocessor}
\label{chap:preprocessor}

The TypeScript-based library introduced in \autoref{chap:library} contains an implementation of primitive parsers as well as commonly used combinators.
However, in order to retain the intuitive syntax that users are likely to be accustomed to from other combinator libraries (in languages with user-defined operators), a preprocessing step is required.
While the introduction of the preprocessor begins to blur the lines between parser combinators and parser generators, it provides an excellent opportunity to perform deeper inspection on parsers.

This chapter begins by introducing the overall flow of data through the preprocessing pipeline.
Throughout the sections, various design decisions and underlying implementation details are explored.
This includes the parser for the augmented grammar, the representation used throughout the optimiser, as well as the aforementioned additions to TypeScript.

\section{Pipeline Architecture}
\label{sec:pipeline_arch}

This section aims to provide a high-level overview of the main processing pipeline, from ingesting the source file, to optimising the program, and back to a source file.
As shown in \autoref{fig:pipeline_diagram}, it is important to note that the optimiser itself consists of multiple stages, which can be partitioned into one of three categories; before the conversion to the IR, while the AST is in IR form, and after the conversion from the IR.
The bulk of the optimisation is performed in the IR stage, whereas the steps before and after are primarily to support additional language features.

\begin{figure}[H]
    \centering
    \begin{tikzpicture}[every node/.style={execute at end node=\vphantom{bg}}]
        \begin{scope}[shift={(0, 0)}]
            \node at (0, 0) {\texttt{F[.tsp] -> AST}};
            \node at (0, 1) {\tsext Parser};
            \draw (-2, 0.5) -- (2, 0.5) -- (2, -0.5) -- (-2, -0.5) -- cycle;
        \end{scope}
        \begin{scope}[shift={(5, 0)}]
            \node at (0, 0) {\texttt{AST -> AST}};
            \node at (0, 1) {Optimiser};
            \draw[dotted] (-2, 0.5) -- (2, 0.5) -- (2, -0.5) -- (-2, -0.5) -- cycle;
        \end{scope}
        \begin{scope}[shift={(10, 0)}]
            \node at (0, 0) {\texttt{AST -> F[.ts]}};
            \node at (0, 1) {AST Printer};
            \draw (-2, 0.5) -- (2, 0.5) -- (2, -0.5) -- (-2, -0.5) -- cycle;
        \end{scope}
        \begin{scope}[shift={(0, -2)}]
            \node at (0, 0) {\texttt{AST -> AST}};
            \node at (0, -1) {Inlining / Implicits};
            \draw (-1.5, 0.5) -- (1.5, 0.5) -- (1.5, -0.5) -- (-1.5, -0.5) -- cycle;
        \end{scope}
        \begin{scope}[shift={(5, -2)}]
            \node at (0, 0) {\texttt{IR -> IR}};
            \node at (0, -1) {Parser Optimisation};
            \draw (-1.5, 0.5) -- (1.5, 0.5) -- (1.5, -0.5) -- (-1.5, -0.5) -- cycle;
        \end{scope}
        \begin{scope}[shift={(10, -2)}]
            \node at (0, 0) {\texttt{AST -> AST}};
            \node at (0, -1) {Language Extension};
            \draw (-1.5, 0.5) -- (1.5, 0.5) -- (1.5, -0.5) -- (-1.5, -0.5) -- cycle;
        \end{scope}
        \draw
        (2, 0) edge[->] (3, 0)
        (7, 0) edge[->] (8, 0)
        (1.5, -2) edge[->, above] node{\tiny \texttt{AST -> IR}} (3.5, -2)
        (6.5, -2) edge[->, above] node{\tiny \texttt{IR -> AST}} (8.5, -2);

        \draw[ultra thick, decorate, decoration={calligraphic brace, amplitude=7pt, raise=-2pt}] (-1.5, -1) -- (11.5, -1);
    \end{tikzpicture}
    \vspace{-0.5\baselineskip}
    \caption{Simplified view of the main stages within the preprocessor pipeline, \texttt{F[.x]} denotes a file with extension \texttt{.x}}
    \label{fig:pipeline_diagram}
\end{figure}

The primary entry point for the preprocessor uses \textit{scopt} \cite{scopt} to parse command-line options (see \autoref{tab:cli_opt}), which allows for certain optimisation stages to be enabled / disabled.
Once a file is read in, with the path specified by an argument, the raw input is then parsed by a \tsext parser.
Note that \tsext refers to TypeScript with the additional language features augmented onto the existing grammar (listed in \autoref{sec:lang_ext}).

Once the input file has been successfully parsed, the AST is fed through the optimisation pipeline, detailed further in \autoref{chap:oaa}.
Each `stage', excluding language extensions and IR conversions, is either enabled or disabled, depending on the corresponding configuration option.
Every pipeline stage extends the \texttt{PipelineStage} trait, which provides default AST traversal properties, allowing for a stage to only implement the desired functionality on a subset of the AST, but maintaining the guarantee that everything will be traversed.
The latter is important to ensure that certain assumptions can be made about the AST being fed into subsequent stages.
For example, it is generally safe to assume that any \texttt{Expression}s after the first conversion stage will be \texttt{IR} nodes (detailed in \autoref{sec:ir}).
On overriding a \texttt{PipelineStage}, care should be taken to maintain the behaviour from the superclass (unless explicitly not required) in order to main the traversal properties.

Each of the pipeline stages can access a shared mutable state.
However, pipeline execution typically implies an order of execution: stages cannot simply jump forward in the pipeline and back despite this functionality possibly being beneficial.
For example, the preprocessor may need to report an error to the user and include TypeScript code.
However, without the ability to use later pipeline stages (namely a conversion from the IR to the AST), the functionality for generating code will have to be duplicated.
In order to prevent this, the mutable state can be `locked', thus preventing any changes.

The final stage of the pipeline converts the AST back into a raw string by printing out the AST recursively.
Note that this maintains no formatting, such as indent levels - however, the syntax is now fully valid TypeScript.
Optionally, this raw file can then be passed through a formatter, such as \textit{Prettier} \cite{prettier}.

\section{\texorpdfstring{\tsext}{TypeScript<STAR>} Parser}
\label{sec:ts_parser}
The first stage of the processing pipeline involves parsing the raw input file, which contains a valid (both syntactic and semantic) \tsext program.
Note that the remainder of the pipeline assumes semantic correctness, especially with types, as full type verification lies outside the scope of the preprocessor.

The parser within the preprocessor targets ECMAScript 2015 \cite{es2015spec}, augmented with TypeScript-specific additions from early 2016 \cite{tsls}.
Certain existing language features have been excluded, such as (but not limited to) decorators and ambient declarations.
\textit{Parsley} \cite{willis18} for Scala is used for parsing the raw input into an abstract syntax tree (AST).
The use of a parser combinator library not only seemed natural due to the nature of this library and preprocessor, but also provided numerous benefits, such as the single-pass nature of the lexing, parsing, and construction stages.
One disadvantage of using parser combinators is the lack of error recovery - however, as the preprocessor typically expects correct, well-formed code, this is not a major concern.

Note that the AST contains additional language features not present in TypeScript.
These additions are transformed in subsequent pipeline stages into other AST nodes, which represent valid TypeScript syntax.
In order to recover a valid TypeScript program at the end of the pipeline, each AST node requires a \texttt{print} function, which recursively converts the tree back into program code.
At the end of the pipeline, this is called on a \texttt{Module} representing the entire program, which is then optionally passed into a code formatter.

The entirety of the preprocessing pipeline, bar the formatter which exists externally, is implemented in Scala, in order to leverage its pattern matching abilities, as well as the portability of the JVM (Java Virtual Machine).
The former allows for significantly simpler analysis over the AST, which is hugely beneficial for optimisations on a tree.

\section{Intermediate Representation}
\label{sec:ir}

In lieu of duplicating significant fragments of code to create an entirely separate intermediate representation (IR) to represent the AST, the IR instead extends the existing \texttt{Expression}s, by generating a mapping between the IR and the original or generated \texttt{Expression}.
The inheritance structure of the IR can be seen in \autoref{fig:ir_inheritance} - note that \texttt{Id} also extends \texttt{BindingPattern} and \texttt{PropertyDefinition} from the original TypeScript AST.
By augmenting the existing AST, a separate representation for other parts of the AST, such as statements or declarations, is not required.

\begin{figure}[H]
    \centering
    \begin{tikzpicture}[every node/.style={execute at end node=\vphantom{bg}}]
        \node[black!50] at (0, 0) (atom) {\texttt{Atom}};
        \node[black!50, right=of atom] (id) {\texttt{Id}};
        \node[right=of id] (comb) {\texttt{Combinator}};
        \node[right=of comb] (func) {\texttt{Func}};
        \node[black!50, right=of func] (irli) {\texttt{IRLiteral}};
        \node[right=of irli] (gene) {\texttt{Generated}};
        \node (p0) at ($(atom.west)!0.5!(gene.east)$) {};
        \node[] (spec) at (gene |- 0, -1) {\texttt{Special}};
        \node[left=of spec] (gf) {\texttt{GeneratedFunc}};
        \node[] (ir) at (p0 |- 0, 1) {\texttt{IR}};
        \node[] (expr) at (p0 |- 0, 2) {\itshape \texttt{Expression}};
        \draw
        (expr) edge[->] (ir)
        (ir) edge[->, bend right=15] (atom)
        (ir) edge[->, bend right=10] (id)
        (ir) edge[->, bend right=5] (comb)
        (ir) edge[->, bend left=5] (func)
        (ir) edge[->, bend left=10] (irli)
        (ir) edge[->, bend left=15] (gene)
        (gene) edge[->] (spec)
        (gene) edge[->, bend right=5] (gf);
    \end{tikzpicture}
    \caption{Inheritance hierarchy within the IR, nodes in \textcolor{black!50}{\texttt{grey}} are concrete whereas nodes in \texttt{black} are traits}
    \label{fig:ir_inheritance}
\end{figure}

\texttt{Atom}s contain a unique numeric identifier which provides a mapping from an \texttt{Atom} to an \texttt{Expression}, which is stored in a mutable state (implicitly used by all pipeline stages).
These contain any \texttt{Expression}s that cannot be changed into any other representation.
Generally, they contain nodes that require no further inspection, as they are unlikely to contain any combinators - especially not ones that can be meaningfully inspected.

On the other hand, \texttt{Combinator}s are a special case of regular binary operators which contain combinators (any other operation will be converted into an \texttt{Atom}).
For each different combinator, there is a different concrete implementation (\texttt{case class}), allowing for easier inspection in subsequent pipeline stages (via matching on a specific combinator, rather than checking which operation is being performed manually).
Each of these combinators has two arguments, both of which are also IRs.
Similarly, \texttt{Func}s have concrete implementations for reserved functions that are provided with the TypeScript library; these have specific arguments which are assumed to be valid when processing.
This adds the ability to perform some inspection on functions which have known behaviours.

\texttt{Id} and \texttt{IRLiteral} both have similar functionality in that they wrap existing nodes (namely, \texttt{Identifier} and \texttt{Literal}) to work within the IR.
The former is handled separately as it can often be useful to be able to directly access variable names in later stages without the need to reference the shared state (if they were instead represented as \texttt{Atom}s).
On the other hand, \texttt{IRLiteral}s are used for a similar reason, however the main purpose is to allow for inspection into string or character literals, which is useful for subsequent stages.

Finally, the \texttt{Generated} trait refers to any nodes without a mapping from an \texttt{Expression}, meaning that they are only generated by the processing pipeline.
This is further divided into one of two traits.
Nodes which extend the \texttt{GeneratedFunc} trait represent functions that are applied to IR nodes.
These functions have a `fallback' implementation within the library; however, the preprocessor will attempt to directly alter the AST whenever possible.
In contrast, nodes which extend the \texttt{Special} trait are typically used to represent specific values or are used to directly pass information to other stages.
For example, it may be required to explicitly pass an \texttt{Expression} forward without any modifications or highlight characters which exist in the first set of a parser.
\bigskip

The example of addition from \autoref{lst:running_example}, after being parsed into the AST shown in \autoref{lst:running_ast}, is translated into the IR as shown in \autoref{lst:running_ir}.
Notice how only the information that is directly related to parsers is maintained, whereas the function is simply converted into \texttt{Atom(0)}.

\begin{capminted}
    \begin{minted}{scala}
        (Id("expr") <**> (Atom(0) <$ Chr(IRLiteral("'+'"))) <*> Id("nat")) <|>
        Id("nat")
    \end{minted}
    \vspace{-0.5\baselineskip}
    \caption{IR of right-hand side of assignment}
    \label{lst:running_ir}
\end{capminted}

\subsection{Function Rewriting}
\label{ssec:func_rewrite}
An example of the function rewriting is as follows, where \texttt{f0} is a function without AST rewriting (uses library functions) and \texttt{f1} is a function that is directly rewritten.
While this may seem like a pathological example, these transformations are used extensively when rewriting the parsers to support left-recursion.

\begin{capminted}
    \begin{minted}{typescript}
        let f0 = flip(curry(unpairFirstArg(
            ([x, y]: [number, number], z: number) => x + y - z
        )));
        let f1 = ([y, z]: [number, number]) => (x: number) => x + y - z;
    \end{minted}
    \vspace{-0.5\baselineskip}
    \caption{Example of function rewriting performed by the preprocessor}
    \label{lst:rewrite_xyz}
\end{capminted}

Not only does the direct rewrite improve code clarity by omitting multiple function calls, there is also a small but measurable performance uplift stemming from the reduced number of invocations.
This can be seen in \autoref{fig:rewrite}, which performs $10,000$ iterations of the function in \autoref{lst:rewrite_xyz}.
Within each iteration, the function is transformed and executed $n$ times.
This result demonstrates that with repeated invocations, the performance gap between the two forms becomes narrower.
However, the directly rewritten form consistently performs better, especially with few invocations when it can complete execution in approximately half the time.

\begin{figure}[H]
    \centering
    \import{figs}{figs/gen_func_rewrite.pgf}
    \vspace{-0.5\baselineskip}
    \caption{Change in average execution time based on number of repetitions}
    \label{fig:rewrite}
\end{figure}

\section{Language Extensions}
\label{sec:lang_ext}

\def\mplw{0.49\textwidth}
\def\mprw{0.49\textwidth}

As part of the preprocessor, a number of languages features are implemented that augment the existing TypeScript grammar, most of which are implemented to provide `syntactic sugar' for writing parser combinators.
This section introduces these additions alongside some context surrounding usage.
Note that for all the subsequent code snippets in this section, the snippet on the left refers to the augmented grammar (\texttt{.tsp}) and the snippet on the right refers to the transpiled TypeScript (\texttt{.ts}).

\subsection{Declarations}
Two new lexical declarations have been added, namely \texttt{val} and \texttt{lazy}.
The former is simply an alias for \texttt{const}, with no additional functionality.
On the other hand, \texttt{lazy} allows for rudimentary laziness, thus allowing for parsers to be recursively defined - required for supporting recursive syntax.
This can be seen in the example below, where \texttt{p} references itself.
Without the additional wrapping for laziness, the code would be semantically invalid as a variable would be used prior to its assignment.

\begin{center}
    \begin{minipage}[t]{\mplw}
        \begin{minted}{teaspoon_lex.py:TeaspoonLexer -x}
            lazy p = p;
            val q = x;
        \end{minted}
    \end{minipage}
    \hfill
    \tikzmark{lt_d}
    \hfill
    \begin{minipage}[t]{\mprw}
        \begin{minted}{typescript}
            const p = lazy(() => p);
            const q = x;
        \end{minted}
    \end{minipage}
\end{center}
\tikz[remember picture] \node[overlay] at ($(pic cs:lt_d) + (-0.5, -0.25)$) {$\leadsto$};

\subsection{User-Defined Operators}
In order to provide an intuitively legible syntax for writing parsers, user-defined operators have been implemented.
Note that the snippet below does not provide usage for all implemented operators, but rather demonstrates how they are supported.
In the example, \texttt{p} is an example of how a simple combinator is written into a function call.
The left-associative property of combinators is also respected, as shown in \texttt{q}, with support for parentheses shown in \texttt{s}.
Similarly, the lower precedence of the choice combinator (\texttt{\choice}) is shown in \texttt{r}.
Finally, arbitrary user-defined functions can be supported, as shown in \texttt{t} - this addition allows for combinators to be defined beyond what is provided by the library, without requiring changes to the preprocessor.

\begin{center}
    \begin{minipage}[t]{\mplw}
        \begin{minted}{teaspoon_lex.py:TeaspoonLexer -x}
            let p = a <~> b;
            let q = a *> b *> c;
            let r =
              a <*> b <|> c <*> d;
            let s =
              a <**> (pf <* b) <*> c;
            let t = a <xyz> b;
        \end{minted}
    \end{minipage}
    \hfill
    \tikzmark{lt_udo}
    \hfill
    \begin{minipage}[t]{\mprw}
        \begin{minted}{typescript}
            let p = mult(q, r);
            let q = apR(apR(a, b), c);
            let r =
              choice(ap(a, b), ap(c, d));
            let s =
              ap(pa(a, apL(pf, b)), c);
            let t = xyz(a, b);
        \end{minted}
    \end{minipage}
\end{center}
\tikz[remember picture] \node[overlay] at ($(pic cs:lt_udo) + (-0.5, -1.375)$) {$\leadsto$};

\subsection{Implicit Conversions}
In order to reduce the amount of conversion the user has to manually perform, implicit conversions can be enabled in the pipeline.
With this enabled, a use of a string, character, or regular expression literal in place of where a parser is expected will cause the preprocessor to automatically apply a conversion (wrapping the function with \texttt{chr}, \texttt{str}, or \texttt{re}, depending on the type of the literal).
This addition provides some ability to have `overloaded strings', as described by Willis \& Wu (2021) \cite{willis21}.

\begin{center}
    \begin{minipage}[t]{\mplw}
        \begin{minted}{teaspoon_lex.py:TeaspoonLexer -x}
            let p = f <*> 'a' <*> /b/i;
            let q = 'c' $> 'd';
            let r = attempt('ef');
        \end{minted}
    \end{minipage}
    \hfill
    \tikzmark{lt_ic}
    \hfill
    \begin{minipage}[t]{\mprw}
        \begin{minted}{typescript}
        let p =
          p(ap(f, chr('a')), re(/b/i));
        let q =
          constFmapR(chr('c'), 'd');
        let r = attempt(str('ef'));
        \end{minted}
    \end{minipage}
\end{center}
\tikz[remember picture] \node[overlay] at ($(pic cs:lt_ic) + (-0.5, -0.85)$) {$\leadsto$};

\subsection{`Macros'}
\label{ssec:inline}
One limitation of the preprocessor, when performing analysis, is observing the behaviour of more `complex' functions or declarations.
While it may be feasible to inline some variable declarations, if they are known to be immutable, it can often negatively impact the readability of the produced code.
It may not be desirable to inline a complex declaration.
On the other hand, it is likely difficult to analyse an arbitrary function, even if the underlying semantics are fairly simple.
For example, consider the function \texttt{between}, provided in the library.
The invocation \texttt{between(l, p, r)} simply denotes \texttt{l \apr p \apl r} - in this case, the function is simply shorthand.
For these cases, the user should be able to specify when a replacement can be safely performed, via the use of the \texttt{inline} `declaration'.

\begin{center}
    \begin{minipage}[t]{\mplw}
        \begin{minted}{teaspoon_lex.py:TeaspoonLexer -x}
            inline paap(x, y, z) =
              x <**> y <*> z;
            inline o = add <$ chr('+');
            inline add = (x: number) =>
              (y: number) => x + y;

            let r = paap(a, o, b);
        \end{minted}
    \end{minipage}
    \hfill
    \tikzmark{lt_m}
    \hfill
    \begin{minipage}[t]{\mprw}
        \begin{minted}{typescript}
            let r = ap(pa(a,
              constFmapL(
                (x: number) =>
                  (y: number) => x + y,
                chr('+'))
            ), b);
        \end{minted}
    \end{minipage}
\end{center}
\tikz[remember picture] \node[overlay] at ($(pic cs:lt_m) + (-0.5, -1.375)$) {$\leadsto$};

The snippet demonstrates how a substitution is performed, for substitutions with and without parameters.
Notice that the substitution is not restricted to only parsers, as seen with \texttt{add}.
It is important to note that this does not perform a na\"ive text replacement, but rather recursively splices in parts of the AST.
By modifying the AST, further analysis can be performed in subsequent stages.
For example, rather than seeing \texttt{r} as being an arbitrary function call, which may have behaviour that cannot trivially be inspected, it is seen as a combinator tree which can be analysed further.
The ability to provide these `hints' to the preprocessor is beneficial to the efficacy of optimisations that analyse combinator trees.

Consider the earlier example of \texttt{between}.
Note that \texttt{p = between(l, a, r)} and \texttt{q = between(l, b, r)} follow a similar form.
However, without knowing how \texttt{between} is defined, it is not trivial to deduce that both \texttt{p} and \texttt{q} share a common prefix.
On the other hand, if both were expanded based on the inline definition, it is trivial to see that both parsers have a common prefix and suffix of \texttt{l} and \texttt{r}, respectively.


\section*{Summary}
This chapter demonstrates how the raw \tsext source code is transformed in order to support numerous language features, including the user-defined operators introduced in \autoref{chap:library}.
The additions to TypeScript are also presented in the context of parser combinators and how a user may find them beneficial.
A deeper dive was also made into the underlying design of the preprocessor itself as well as how the parsed structure is traversed and simplified for optimisations.

However, the theory and implementation of the optimisation steps have been omitted for brevity.
Instead, they are further explored in \autoref{chap:oaa}, which dives deeper into the techniques applied and how they transform the grammar.
