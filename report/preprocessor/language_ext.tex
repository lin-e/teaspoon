\section{Language Extensions}
\label{sec:lang_ext}

\def\mplw{0.49\textwidth}
\def\mprw{0.49\textwidth}

As part of the preprocessor, a number of languages features are implemented that augment the existing TypeScript grammar, most of which are implemented to provide `syntactic sugar' for writing parser combinators.
This section introduces these additions alongside some context surrounding usage.
Note that for all the subsequent code snippets in this section, the snippet on the left refers to the augmented grammar (\texttt{.tsp}) and the snippet on the right refers to the transpiled TypeScript (\texttt{.ts}).

\subsection{Declarations}
Two new lexical declarations have been added, namely \texttt{val} and \texttt{lazy}.
The former is simply an alias for \texttt{const}, with no additional functionality.
On the other hand, \texttt{lazy} allows for rudimentary laziness, thus allowing for parsers to be recursively defined - required for supporting recursive syntax.
This can be seen in the example below, where \texttt{p} references itself.
Without the additional wrapping for laziness, the code would be semantically invalid as a variable would be used prior to its assignment.

\begin{center}
    \begin{minipage}[t]{\mplw}
        \begin{minted}{teaspoon_lex.py:TeaspoonLexer -x}
            lazy p = p;
            val q = x;
        \end{minted}
    \end{minipage}
    \hfill
    \tikzmark{lt_d}
    \hfill
    \begin{minipage}[t]{\mprw}
        \begin{minted}{typescript}
            const p = lazy(() => p);
            const q = x;
        \end{minted}
    \end{minipage}
\end{center}
\tikz[remember picture] \node[overlay] at ($(pic cs:lt_d) + (-0.5, -0.25)$) {$\leadsto$};

\subsection{User-Defined Operators}
In order to provide an intuitively legible syntax for writing parsers, user-defined operators have been implemented.
Note that the snippet below does not provide usage for all implemented operators, but rather demonstrates how they are supported.
In the example, \texttt{p} is an example of how a simple combinator is written into a function call.
The left-associative property of combinators is also respected, as shown in \texttt{q}, with support for parentheses shown in \texttt{s}.
Similarly, the lower precedence of the choice combinator (\texttt{\choice}) is shown in \texttt{r}.
Finally, arbitrary user-defined functions can be supported, as shown in \texttt{t} - this addition allows for combinators to be defined beyond what is provided by the library, without requiring changes to the preprocessor.

\begin{center}
    \begin{minipage}[t]{\mplw}
        \begin{minted}{teaspoon_lex.py:TeaspoonLexer -x}
            let p = a <~> b;
            let q = a *> b *> c;
            let r =
              a <*> b <|> c <*> d;
            let s =
              a <**> (pf <* b) <*> c;
            let t = a <xyz> b;
        \end{minted}
    \end{minipage}
    \hfill
    \tikzmark{lt_udo}
    \hfill
    \begin{minipage}[t]{\mprw}
        \begin{minted}{typescript}
            let p = mult(q, r);
            let q = apR(apR(a, b), c);
            let r =
              choice(ap(a, b), ap(c, d));
            let s =
              ap(pa(a, apL(pf, b)), c);
            let t = xyz(a, b);
        \end{minted}
    \end{minipage}
\end{center}
\tikz[remember picture] \node[overlay] at ($(pic cs:lt_udo) + (-0.5, -1.375)$) {$\leadsto$};

\subsection{Implicit Conversions}
In order to reduce the amount of conversion the user has to manually perform, implicit conversions can be enabled in the pipeline.
With this enabled, a use of a string, character, or regular expression literal in place of where a parser is expected will cause the preprocessor to automatically apply a conversion (wrapping the function with \texttt{chr}, \texttt{str}, or \texttt{re}, depending on the type of the literal).
This addition provides some ability to have `overloaded strings', as described by Willis \& Wu (2021) \cite{willis21}.

\begin{center}
    \begin{minipage}[t]{\mplw}
        \begin{minted}{teaspoon_lex.py:TeaspoonLexer -x}
            let p = f <*> 'a' <*> /b/i;
            let q = 'c' $> 'd';
            let r = attempt('ef');
        \end{minted}
    \end{minipage}
    \hfill
    \tikzmark{lt_ic}
    \hfill
    \begin{minipage}[t]{\mprw}
        \begin{minted}{typescript}
        let p =
          p(ap(f, chr('a')), re(/b/i));
        let q =
          constFmapR(chr('c'), 'd');
        let r = attempt(str('ef'));
        \end{minted}
    \end{minipage}
\end{center}
\tikz[remember picture] \node[overlay] at ($(pic cs:lt_ic) + (-0.5, -0.85)$) {$\leadsto$};

\subsection{`Macros'}
\label{ssec:inline}
One limitation of the preprocessor, when performing analysis, is observing the behaviour of more `complex' functions or declarations.
While it may be feasible to inline some variable declarations, if they are known to be immutable, it can often negatively impact the readability of the produced code.
It may not be desirable to inline a complex declaration.
On the other hand, it is likely difficult to analyse an arbitrary function, even if the underlying semantics are fairly simple.
For example, consider the function \texttt{between}, provided in the library.
The invocation \texttt{between(l, p, r)} simply denotes \texttt{l \apr p \apl r} - in this case, the function is simply shorthand.
For these cases, the user should be able to specify when a replacement can be safely performed, via the use of the \texttt{inline} `declaration'.

\begin{center}
    \begin{minipage}[t]{\mplw}
        \begin{minted}{teaspoon_lex.py:TeaspoonLexer -x}
            inline paap(x, y, z) =
              x <**> y <*> z;
            inline o = add <$ chr('+');
            inline add = (x: number) =>
              (y: number) => x + y;

            let r = paap(a, o, b);
        \end{minted}
    \end{minipage}
    \hfill
    \tikzmark{lt_m}
    \hfill
    \begin{minipage}[t]{\mprw}
        \begin{minted}{typescript}
            let r = ap(pa(a,
              constFmapL(
                (x: number) =>
                  (y: number) => x + y,
                chr('+'))
            ), b);
        \end{minted}
    \end{minipage}
\end{center}
\tikz[remember picture] \node[overlay] at ($(pic cs:lt_m) + (-0.5, -1.375)$) {$\leadsto$};

The snippet demonstrates how a substitution is performed, for substitutions with and without parameters.
Notice that the substitution is not restricted to only parsers, as seen with \texttt{add}.
It is important to note that this does not perform a na\"ive text replacement, but rather recursively splices in parts of the AST.
By modifying the AST, further analysis can be performed in subsequent stages.
For example, rather than seeing \texttt{r} as being an arbitrary function call, which may have behaviour that cannot trivially be inspected, it is seen as a combinator tree which can be analysed further.
The ability to provide these `hints' to the preprocessor is beneficial to the efficacy of optimisations that analyse combinator trees.

Consider the earlier example of \texttt{between}.
Note that \texttt{p = between(l, a, r)} and \texttt{q = between(l, b, r)} follow a similar form.
However, without knowing how \texttt{between} is defined, it is not trivial to deduce that both \texttt{p} and \texttt{q} share a common prefix.
On the other hand, if both were expanded based on the inline definition, it is trivial to see that both parsers have a common prefix and suffix of \texttt{l} and \texttt{r}, respectively.
