\section{Pipeline Architecture}
\label{sec:pipeline_arch}

This section aims to provide a high-level overview of the main processing pipeline, from ingesting the source file, to optimising the program, and back to a source file.
As shown in \autoref{fig:pipeline_diagram}, it is important to note that the optimiser itself consists of multiple stages, which can be partitioned into one of three categories; before the conversion to the IR, while the AST is in IR form, and after the conversion from the IR.
The bulk of the optimisation is performed in the IR stage, whereas the steps before and after are primarily to support additional language features.

\begin{figure}[H]
    \centering
    \begin{tikzpicture}[every node/.style={execute at end node=\vphantom{bg}}]
        \begin{scope}[shift={(0, 0)}]
            \node at (0, 0) {\texttt{F[.tsp] -> AST}};
            \node at (0, 1) {\tsext Parser};
            \draw (-2, 0.5) -- (2, 0.5) -- (2, -0.5) -- (-2, -0.5) -- cycle;
        \end{scope}
        \begin{scope}[shift={(5, 0)}]
            \node at (0, 0) {\texttt{AST -> AST}};
            \node at (0, 1) {Optimiser};
            \draw[dotted] (-2, 0.5) -- (2, 0.5) -- (2, -0.5) -- (-2, -0.5) -- cycle;
        \end{scope}
        \begin{scope}[shift={(10, 0)}]
            \node at (0, 0) {\texttt{AST -> F[.ts]}};
            \node at (0, 1) {AST Printer};
            \draw (-2, 0.5) -- (2, 0.5) -- (2, -0.5) -- (-2, -0.5) -- cycle;
        \end{scope}
        \begin{scope}[shift={(0, -2)}]
            \node at (0, 0) {\texttt{AST -> AST}};
            \node at (0, -1) {Inlining / Implicits};
            \draw (-1.5, 0.5) -- (1.5, 0.5) -- (1.5, -0.5) -- (-1.5, -0.5) -- cycle;
        \end{scope}
        \begin{scope}[shift={(5, -2)}]
            \node at (0, 0) {\texttt{IR -> IR}};
            \node at (0, -1) {Parser Optimisation};
            \draw (-1.5, 0.5) -- (1.5, 0.5) -- (1.5, -0.5) -- (-1.5, -0.5) -- cycle;
        \end{scope}
        \begin{scope}[shift={(10, -2)}]
            \node at (0, 0) {\texttt{AST -> AST}};
            \node at (0, -1) {Language Extension};
            \draw (-1.5, 0.5) -- (1.5, 0.5) -- (1.5, -0.5) -- (-1.5, -0.5) -- cycle;
        \end{scope}
        \draw
        (2, 0) edge[->] (3, 0)
        (7, 0) edge[->] (8, 0)
        (1.5, -2) edge[->, above] node{\tiny \texttt{AST -> IR}} (3.5, -2)
        (6.5, -2) edge[->, above] node{\tiny \texttt{IR -> AST}} (8.5, -2);

        \draw[ultra thick, decorate, decoration={calligraphic brace, amplitude=7pt, raise=-2pt}] (-1.5, -1) -- (11.5, -1);
    \end{tikzpicture}
    \vspace{-0.5\baselineskip}
    \caption{Simplified view of the main stages within the preprocessor pipeline, \texttt{F[.x]} denotes a file with extension \texttt{.x}}
    \label{fig:pipeline_diagram}
\end{figure}

The primary entry point for the preprocessor uses \textit{scopt} \cite{scopt} to parse command-line options (see \autoref{tab:cli_opt}), which allows for certain optimisation stages to be enabled / disabled.
Once a file is read in, with the path specified by an argument, the raw input is then parsed by a \tsext parser.
Note that \tsext refers to TypeScript with the additional language features augmented onto the existing grammar (listed in \autoref{sec:lang_ext}).

Once the input file has been successfully parsed, the AST is fed through the optimisation pipeline, detailed further in \autoref{chap:oaa}.
Each `stage', excluding language extensions and IR conversions, is either enabled or disabled, depending on the corresponding configuration option.
Every pipeline stage extends the \texttt{PipelineStage} trait, which provides default AST traversal properties, allowing for a stage to only implement the desired functionality on a subset of the AST, but maintaining the guarantee that everything will be traversed.
The latter is important to ensure that certain assumptions can be made about the AST being fed into subsequent stages.
For example, it is generally safe to assume that any \texttt{Expression}s after the first conversion stage will be \texttt{IR} nodes (detailed in \autoref{sec:ir}).
On overriding a \texttt{PipelineStage}, care should be taken to maintain the behaviour from the superclass (unless explicitly not required) in order to main the traversal properties.

Each of the pipeline stages can access a shared mutable state.
However, pipeline execution typically implies an order of execution: stages cannot simply jump forward in the pipeline and back despite this functionality possibly being beneficial.
For example, the preprocessor may need to report an error to the user and include TypeScript code.
However, without the ability to use later pipeline stages (namely a conversion from the IR to the AST), the functionality for generating code will have to be duplicated.
In order to prevent this, the mutable state can be `locked', thus preventing any changes.

The final stage of the pipeline converts the AST back into a raw string by printing out the AST recursively.
Note that this maintains no formatting, such as indent levels - however, the syntax is now fully valid TypeScript.
Optionally, this raw file can then be passed through a formatter, such as \textit{Prettier} \cite{prettier}.
