\section{Normalisation}
\label{sec:sound_normalisation}

Similar to the reduction rules, the normalisation rules are used within the left recursion rewriting process.
Validity is shown for the normalisation steps, as well as for any function transformations that occur.

\subsection{Normalisation of \texttt{p \pa (f \constfmapl q) \ap r}}
Note that unlike the other cases, this case contains an additional discarded parser (\texttt{q}).
The goal is to show that the normalisation of $\texttt{a} = \texttt{p \pa (f \constfmapl q) \ap r}$ to $\texttt{b} = \texttt{lift2(f}_\texttt{u}\texttt{, p, q \apr r)}$ is sound.

\begin{align*}
    \texttt{a} =\ & \texttt{pure (flip (\$)) \ap p \ap (f \constfmapl q) \ap r} & \equi[1] \\
    =\ & \texttt{pure (flip (\$)) \ap p \ap (pure (const f) \ap q) \ap r} & \equi[13] \\
    =\ & \texttt{pure (.) \ap (pure (flip (\$)) \ap p) \ap} & \equi[4] \\
    & \texttt{pure (const f) \ap q \ap r} \\
    =\ & \texttt{pure (.).(flip (\$)) \ap p \ap pure (const f) \ap q \ap r} & \equi[7] \\
    =\ & \texttt{pure }\overbrace{\texttt{(\textbackslash g -> g (const f)).((.).(flip (\$)))}}^{\alpha}\texttt{ \ap p \ap} & \equi[8] \\
    & \texttt{q \ap r} \\
    \texttt{b} =\ & \texttt{pure }\underbrace{\texttt{(\textbackslash g -> g FC).((.).((.).f))}}_{\beta}\texttt{ \ap p \ap q \ap r} & \text{(*)} \\
    \alpha =\ & (\lambda g \to g\ ((\lambda pq \to p)\ f)) \cdot ((\cdot) \cdot (\lambda xh \to h\ x)) \\
    =\ & (\lambda g \to g\ (\lambda q \to f)) \cdot ((\cdot) \cdot (\lambda xh \to h\ x)) \\
    =\ & \lambda x \to (\lambda g \to g\ (\lambda q \to f)) (((\cdot) \cdot (\lambda xh \to h\ x))\ x) \\
    =\ & \lambda x \to (\lambda g \to g\ (\lambda q \to f)) ((\cdot)(\lambda h \to h\ x)) \\
    =\ & \lambda x \to (\lambda h \to h\ x) \cdot (\lambda q \to f) \\
    =\ & \lambda xy \to (\lambda h \to h\ x)\ f \\
    =\ & \lambda xy \to f\ x \\
    =\ & \lambda xyz \to f\ x\ z \\
    \beta =\ & \lambda xyz \to f\ x\ z & \text{(*)}
\end{align*}
Note that the results for \texttt{b} and $\beta$ are reused from \autoref{ssec:reduce_discard} for brevity.
Using these results, $\alpha = \beta$, therefore $\texttt{a} = \texttt{b}$ - hence the normalisation is sound.

\subsection{Normalisation of \texorpdfstring{\texttt{f \fmap (p \mult q)}}{F<fmap>(P<mult>Q)}}
The goal is to show that normalising $\texttt{a} = \texttt{f \fmap (p \mult q)}$ to $\texttt{b} = \texttt{lift2(g}_\texttt{u}\texttt{, p, q)}$ is sound.
\begin{align*}
    \texttt{a} =\ & \texttt{pure f \ap (p \mult q)} & \equi[5] \\
    =\ & \texttt{pure f \ap (pure (,) \ap p \ap q)} & \equi[11] \\
    =\ & \texttt{pure (.) \ap pure f \ap (pure (,) \ap p) \ap q} & \equi[4] \\
    =\ & \texttt{pure f. \ap (pure (,) \ap p) \ap q} & \equi[3] \\
    =\ & \texttt{pure }\underbrace{\texttt{(f.).(,)}}_{\alpha}\texttt{ \ap p \ap q} & \equi[7] \\
    \alpha =\ & (f \cdot) \cdot (,) \\
    =\ & \lambda x \to (f \cdot) (x,) \\
    =\ & \lambda xy \to f\ (x, y)
\end{align*}

The function \texttt{g} can trivially be constructed as one that takes in two arguments and combines them into a tuple, which is then passed into \texttt{f}, as shown in $\alpha$.

\subsection{Normalisation of \texttt{p \pa pure f \ap q}}
The goal is to show that a normalisation step from $\texttt{a} = \texttt{p \pa pure f \ap q}$ to $\texttt{b} = \texttt{lift2(f}_\texttt{u}\texttt{, p, q)}$ is sound.
\begin{align*}
    \texttt{a} =\ & \texttt{pure (flip (\$)) \ap p \ap pure f \ap q} & \equi[1] \\
    =\ & \texttt{pure }\underbrace{\texttt{(\textbackslash g -> g f).(flip (\$))}}_{\alpha}\texttt{ \ap p \ap q} & \equi[8] \\
    \alpha =\ & (\lambda g \to g\ f) \cdot (\lambda xf \to f\ x) \\
    =\ & \lambda x \to (\lambda g \to g\ f)(\lambda f \to f\ x) \\
    =\ & \lambda x \to (\lambda f \to f\ x)\ f \\
    =\ & \lambda x \to f\ x
\end{align*}

As $\alpha$ can be shown to be equivalent to \texttt{f}, it follows that $\texttt{a} = \texttt{b}$, hence the normalisation step is valid.
